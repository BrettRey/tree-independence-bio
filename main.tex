\documentclass[12pt]{article}

% House style preamble
% !TEX TS-program = xelatex
% File: preamble.tex
% Purpose: Brett Reynolds house style LaTeX preamble
% Version: 2.0.0 (December 2025 typography redesign)
%
% Usage: % !TEX TS-program = xelatex
% File: preamble.tex
% Purpose: Brett Reynolds house style LaTeX preamble
% Version: 2.0.0 (December 2025 typography redesign)
%
% Usage: % !TEX TS-program = xelatex
% File: preamble.tex
% Purpose: Brett Reynolds house style LaTeX preamble
% Version: 2.0.0 (December 2025 typography redesign)
%
% Usage: \input{.house-style/preamble.tex} in main document
%
% Compiler: XeLaTeX (default) or LuaLaTeX
%           NOT pdfLaTeX - requires fontspec for fonts

% ===========================
% FONTS
% ===========================
\usepackage{fontspec}

\setmainfont{EB Garamond}[
  Numbers=OldStyle,
  Ligatures=TeX,
]

% IPA fallback font
\newfontfamily\ipafont{Charis SIL}
\newcommand{\ipa}[1]{{\ipafont #1}}

% Lining figures when needed (tables, years in isolation)
\providecommand{\liningnums}[1]{{\addfontfeatures{Numbers=Lining}#1}}

% Monospace
\setmonofont{Inconsolata}[Scale=MatchLowercase]

% ===========================
% PAGE LAYOUT
% ===========================
\usepackage[
  letterpaper,
  inner=1.25in,
  outer=1in,
  top=1in,
  bottom=1.25in,
  marginparwidth=0.6in,
]{geometry}

\usepackage[british]{babel}
\usepackage[final]{microtype}

% ===========================
% HEADINGS
% ===========================
\usepackage{titlesec}

% Section: small caps, number in margin
\titleformat{\section}
  {\normalfont\scshape}
  {\llap{\thesection\quad}}
  {0pt}
  {}

% Subsection: small caps sentence case
\titleformat{\subsection}
  {\normalfont\scshape}
  {\thesubsection\quad}
  {0pt}
  {}

% Subsubsection: upright
\titleformat{\subsubsection}
  {\normalfont}
  {\thesubsubsection\quad}
  {0pt}
  {}

% Spacing
\titlespacing*{\section}{0pt}{2ex plus 1ex minus .2ex}{1ex plus .2ex}
\titlespacing*{\subsection}{0pt}{1.5ex plus 1ex minus .2ex}{0.5ex plus .2ex}
\titlespacing*{\subsubsection}{0pt}{1ex plus 0.5ex minus .1ex}{0.3ex plus .1ex}

% ===========================
% RUNNING HEADS
% ===========================
\usepackage{fancyhdr}
\pagestyle{fancy}
\fancyhf{}
\fancyhead[LE]{\small\scshape\leftmark}
\fancyhead[RO]{\small\scshape\@title}
\fancyfoot[LE,RO]{\thepage}
\renewcommand{\headrulewidth}{0pt}

% For oneside documents (most papers):
\fancyhead[L]{\small\scshape\leftmark}
\fancyhead[R]{\small\thepage}
\fancyfoot{}

% ===========================
% COLORS & HYPERLINKS
% ===========================
\usepackage{xcolor}
\definecolor{linkmaroon}{RGB}{128, 0, 32}

\usepackage{hyperref}
\hypersetup{
  colorlinks=true,
  linkcolor=linkmaroon,
  citecolor=linkmaroon,
  urlcolor=linkmaroon,
  pdfauthor={Brett Reynolds},
}

\usepackage{orcidlink}

% ===========================
% QUOTATIONS
% ===========================
\usepackage[style=american]{csquotes} % Double quotes outer (Oxford spelling kept via babel)
\MakeOuterQuote{"}

% ===========================
% SEMANTIC MACROS
% ===========================
% Terms (linguistic concepts being introduced/defined)
\newcommand{\term}[1]{\textsc{#1}}

% Mentions (words/expressions under discussion)
\newcommand{\mention}[1]{\textit{#1}}

% Mentions in headings (angle brackets instead of italics)
% Robust: headings are moving arguments (written to .toc/.aux)
\DeclareRobustCommand{\mentionhead}[1]{⟨\textup{#1}⟩}

% Object language (cited forms, foreign words)
\newcommand{\olang}[1]{\textit{#1}}

% Small-caps abbreviations for glosses
\newcommand{\abbr}[1]{\textsc{#1}}

% Cross-linguistic subscript marker
\usepackage{marvosym}
\newcommand{\crossmark}{\textsubscript{\Cross}}

% ===========================
% MATHS AND SYMBOLS
% ===========================
\usepackage{amsmath,amssymb}

% ===========================
% LINGUISTIC EXAMPLES
% ===========================
\usepackage{langsci-gb4e}
\makeatletter
\@ifundefined{noautomath}{}{\noautomath}
\makeatother

% Judgement markers
\newcommand{\ungram}[1]{*\!#1}
\newcommand{\marg}[1]{?\!#1}
\newcommand{\odd}[1]{\#\!#1}

% ===========================
% LISTS
% ===========================
\usepackage{enumitem}
\setlist{nosep, leftmargin=*}
\setlist[enumerate]{label=\arabic*.}
\setlist[itemize]{label=--}

% ===========================
% TABLES & FIGURES
% ===========================
\usepackage{booktabs}
\usepackage{array}
\usepackage{graphicx}
\graphicspath{{figures/}}

% ===========================
% BIBLIOGRAPHY
% ===========================
\usepackage[backend=biber,style=apa,natbib=true,doi=true,isbn=false,url=true]{biblatex}
\addbibresource{references.bib}
\IfFileExists{references-local.bib}{\addbibresource{references-local.bib}}{}

% Possessive citation: Author's (Year)
\newcommand{\posscite}[1]{\citeauthor{#1}'s (\citeyear{#1})}

% ===========================
% UTILITIES
% ===========================
\usepackage{xspace}
\newcommand{\eg}{e.g.,\xspace}
\newcommand{\ie}{i.e.,\xspace}
\newcommand{\etc}{etc.\xspace}
 in main document
%
% Compiler: XeLaTeX (default) or LuaLaTeX
%           NOT pdfLaTeX - requires fontspec for fonts

% ===========================
% FONTS
% ===========================
\usepackage{fontspec}

\setmainfont{EB Garamond}[
  Numbers=OldStyle,
  Ligatures=TeX,
]

% IPA fallback font
\newfontfamily\ipafont{Charis SIL}
\newcommand{\ipa}[1]{{\ipafont #1}}

% Lining figures when needed (tables, years in isolation)
\providecommand{\liningnums}[1]{{\addfontfeatures{Numbers=Lining}#1}}

% Monospace
\setmonofont{Inconsolata}[Scale=MatchLowercase]

% ===========================
% PAGE LAYOUT
% ===========================
\usepackage[
  letterpaper,
  inner=1.25in,
  outer=1in,
  top=1in,
  bottom=1.25in,
  marginparwidth=0.6in,
]{geometry}

\usepackage[british]{babel}
\usepackage[final]{microtype}

% ===========================
% HEADINGS
% ===========================
\usepackage{titlesec}

% Section: small caps, number in margin
\titleformat{\section}
  {\normalfont\scshape}
  {\llap{\thesection\quad}}
  {0pt}
  {}

% Subsection: small caps sentence case
\titleformat{\subsection}
  {\normalfont\scshape}
  {\thesubsection\quad}
  {0pt}
  {}

% Subsubsection: upright
\titleformat{\subsubsection}
  {\normalfont}
  {\thesubsubsection\quad}
  {0pt}
  {}

% Spacing
\titlespacing*{\section}{0pt}{2ex plus 1ex minus .2ex}{1ex plus .2ex}
\titlespacing*{\subsection}{0pt}{1.5ex plus 1ex minus .2ex}{0.5ex plus .2ex}
\titlespacing*{\subsubsection}{0pt}{1ex plus 0.5ex minus .1ex}{0.3ex plus .1ex}

% ===========================
% RUNNING HEADS
% ===========================
\usepackage{fancyhdr}
\pagestyle{fancy}
\fancyhf{}
\fancyhead[LE]{\small\scshape\leftmark}
\fancyhead[RO]{\small\scshape\@title}
\fancyfoot[LE,RO]{\thepage}
\renewcommand{\headrulewidth}{0pt}

% For oneside documents (most papers):
\fancyhead[L]{\small\scshape\leftmark}
\fancyhead[R]{\small\thepage}
\fancyfoot{}

% ===========================
% COLORS & HYPERLINKS
% ===========================
\usepackage{xcolor}
\definecolor{linkmaroon}{RGB}{128, 0, 32}

\usepackage{hyperref}
\hypersetup{
  colorlinks=true,
  linkcolor=linkmaroon,
  citecolor=linkmaroon,
  urlcolor=linkmaroon,
  pdfauthor={Brett Reynolds},
}

\usepackage{orcidlink}

% ===========================
% QUOTATIONS
% ===========================
\usepackage[style=american]{csquotes} % Double quotes outer (Oxford spelling kept via babel)
\MakeOuterQuote{"}

% ===========================
% SEMANTIC MACROS
% ===========================
% Terms (linguistic concepts being introduced/defined)
\newcommand{\term}[1]{\textsc{#1}}

% Mentions (words/expressions under discussion)
\newcommand{\mention}[1]{\textit{#1}}

% Mentions in headings (angle brackets instead of italics)
% Robust: headings are moving arguments (written to .toc/.aux)
\DeclareRobustCommand{\mentionhead}[1]{⟨\textup{#1}⟩}

% Object language (cited forms, foreign words)
\newcommand{\olang}[1]{\textit{#1}}

% Small-caps abbreviations for glosses
\newcommand{\abbr}[1]{\textsc{#1}}

% Cross-linguistic subscript marker
\usepackage{marvosym}
\newcommand{\crossmark}{\textsubscript{\Cross}}

% ===========================
% MATHS AND SYMBOLS
% ===========================
\usepackage{amsmath,amssymb}

% ===========================
% LINGUISTIC EXAMPLES
% ===========================
\usepackage{langsci-gb4e}
\makeatletter
\@ifundefined{noautomath}{}{\noautomath}
\makeatother

% Judgement markers
\newcommand{\ungram}[1]{*\!#1}
\newcommand{\marg}[1]{?\!#1}
\newcommand{\odd}[1]{\#\!#1}

% ===========================
% LISTS
% ===========================
\usepackage{enumitem}
\setlist{nosep, leftmargin=*}
\setlist[enumerate]{label=\arabic*.}
\setlist[itemize]{label=--}

% ===========================
% TABLES & FIGURES
% ===========================
\usepackage{booktabs}
\usepackage{array}
\usepackage{graphicx}
\graphicspath{{figures/}}

% ===========================
% BIBLIOGRAPHY
% ===========================
\usepackage[backend=biber,style=apa,natbib=true,doi=true,isbn=false,url=true]{biblatex}
\addbibresource{references.bib}
\IfFileExists{references-local.bib}{\addbibresource{references-local.bib}}{}

% Possessive citation: Author's (Year)
\newcommand{\posscite}[1]{\citeauthor{#1}'s (\citeyear{#1})}

% ===========================
% UTILITIES
% ===========================
\usepackage{xspace}
\newcommand{\eg}{e.g.,\xspace}
\newcommand{\ie}{i.e.,\xspace}
\newcommand{\etc}{etc.\xspace}
 in main document
%
% Compiler: XeLaTeX (default) or LuaLaTeX
%           NOT pdfLaTeX - requires fontspec for fonts

% ===========================
% FONTS
% ===========================
\usepackage{fontspec}

\setmainfont{EB Garamond}[
  Numbers=OldStyle,
  Ligatures=TeX,
]

% IPA fallback font
\newfontfamily\ipafont{Charis SIL}
\newcommand{\ipa}[1]{{\ipafont #1}}

% Lining figures when needed (tables, years in isolation)
\providecommand{\liningnums}[1]{{\addfontfeatures{Numbers=Lining}#1}}

% Monospace
\setmonofont{Inconsolata}[Scale=MatchLowercase]

% ===========================
% PAGE LAYOUT
% ===========================
\usepackage[
  letterpaper,
  inner=1.25in,
  outer=1in,
  top=1in,
  bottom=1.25in,
  marginparwidth=0.6in,
]{geometry}

\usepackage[british]{babel}
\usepackage[final]{microtype}

% ===========================
% HEADINGS
% ===========================
\usepackage{titlesec}

% Section: small caps, number in margin
\titleformat{\section}
  {\normalfont\scshape}
  {\llap{\thesection\quad}}
  {0pt}
  {}

% Subsection: small caps sentence case
\titleformat{\subsection}
  {\normalfont\scshape}
  {\thesubsection\quad}
  {0pt}
  {}

% Subsubsection: upright
\titleformat{\subsubsection}
  {\normalfont}
  {\thesubsubsection\quad}
  {0pt}
  {}

% Spacing
\titlespacing*{\section}{0pt}{2ex plus 1ex minus .2ex}{1ex plus .2ex}
\titlespacing*{\subsection}{0pt}{1.5ex plus 1ex minus .2ex}{0.5ex plus .2ex}
\titlespacing*{\subsubsection}{0pt}{1ex plus 0.5ex minus .1ex}{0.3ex plus .1ex}

% ===========================
% RUNNING HEADS
% ===========================
\usepackage{fancyhdr}
\pagestyle{fancy}
\fancyhf{}
\fancyhead[LE]{\small\scshape\leftmark}
\fancyhead[RO]{\small\scshape\@title}
\fancyfoot[LE,RO]{\thepage}
\renewcommand{\headrulewidth}{0pt}

% For oneside documents (most papers):
\fancyhead[L]{\small\scshape\leftmark}
\fancyhead[R]{\small\thepage}
\fancyfoot{}

% ===========================
% COLORS & HYPERLINKS
% ===========================
\usepackage{xcolor}
\definecolor{linkmaroon}{RGB}{128, 0, 32}

\usepackage{hyperref}
\hypersetup{
  colorlinks=true,
  linkcolor=linkmaroon,
  citecolor=linkmaroon,
  urlcolor=linkmaroon,
  pdfauthor={Brett Reynolds},
}

\usepackage{orcidlink}

% ===========================
% QUOTATIONS
% ===========================
\usepackage[style=american]{csquotes} % Double quotes outer (Oxford spelling kept via babel)
\MakeOuterQuote{"}

% ===========================
% SEMANTIC MACROS
% ===========================
% Terms (linguistic concepts being introduced/defined)
\newcommand{\term}[1]{\textsc{#1}}

% Mentions (words/expressions under discussion)
\newcommand{\mention}[1]{\textit{#1}}

% Mentions in headings (angle brackets instead of italics)
% Robust: headings are moving arguments (written to .toc/.aux)
\DeclareRobustCommand{\mentionhead}[1]{⟨\textup{#1}⟩}

% Object language (cited forms, foreign words)
\newcommand{\olang}[1]{\textit{#1}}

% Small-caps abbreviations for glosses
\newcommand{\abbr}[1]{\textsc{#1}}

% Cross-linguistic subscript marker
\usepackage{marvosym}
\newcommand{\crossmark}{\textsubscript{\Cross}}

% ===========================
% MATHS AND SYMBOLS
% ===========================
\usepackage{amsmath,amssymb}

% ===========================
% LINGUISTIC EXAMPLES
% ===========================
\usepackage{langsci-gb4e}
\makeatletter
\@ifundefined{noautomath}{}{\noautomath}
\makeatother

% Judgement markers
\newcommand{\ungram}[1]{*\!#1}
\newcommand{\marg}[1]{?\!#1}
\newcommand{\odd}[1]{\#\!#1}

% ===========================
% LISTS
% ===========================
\usepackage{enumitem}
\setlist{nosep, leftmargin=*}
\setlist[enumerate]{label=\arabic*.}
\setlist[itemize]{label=--}

% ===========================
% TABLES & FIGURES
% ===========================
\usepackage{booktabs}
\usepackage{array}
\usepackage{graphicx}
\graphicspath{{figures/}}

% ===========================
% BIBLIOGRAPHY
% ===========================
\usepackage[backend=biber,style=apa,natbib=true,doi=true,isbn=false,url=true]{biblatex}
\addbibresource{references.bib}
\IfFileExists{references-local.bib}{\addbibresource{references-local.bib}}{}

% Possessive citation: Author's (Year)
\newcommand{\posscite}[1]{\citeauthor{#1}'s (\citeyear{#1})}

% ===========================
% UTILITIES
% ===========================
\usepackage{xspace}
\newcommand{\eg}{e.g.,\xspace}
\newcommand{\ie}{i.e.,\xspace}
\newcommand{\etc}{etc.\xspace}


% Project-specific macros
\newcommand{\indpoly}{I(T;\,x)}
\newcommand{\nm}{\mathrm{nm}}
\newcommand{\occ}{P}
\DeclareMathOperator{\mode}{mode}

% Update PDF metadata
\hypersetup{
  pdftitle={Tree Independence Polynomials and Biological Network Motifs}
}

\title{Tree Independence Polynomials and Biological Network Motifs}
\author{Brett Reynolds \orcidlink{0000-0003-0073-7195}%
\thanks{Contact: \href{mailto:brett.reynolds@humber.ca}{brett.reynolds@humber.ca}}\\
Humber Polytechnic \& University of Toronto}
\date{\today}

\begin{document}
\maketitle

\begin{abstract}
TODO: Write abstract here.
\end{abstract}

\section{Introduction}

Trees are everywhere in biology. Phylogenies record the branching history of species \citep{hinchliff2015synthesis}. Dendritic arbors carry signals from synapse to soma. The backbone of a transcription regulatory network, stripped of its cycles, is a tree \citep{alon2007network}. In each case the branching pattern isn't just a shape; it constrains what the system can do. This paper asks what a single polynomial can reveal about that constraint.

The \term{independence polynomial} of a tree~$T$ counts the ways to select non-adjacent vertices. Call a set of vertices \term{independent} if no two are neighbours, and let $i_k(T)$ be the number of independent sets of size~$k$. The polynomial $\indpoly = \sum_k i_k(T)\, x^k$ packages these counts. \textcite{alavi1987vertex} conjectured that for every tree the coefficient sequence $i_0, i_1, \ldots$ is \term{unimodal}: it rises to a single peak and then falls. This is Erd\H{o}s Problem~993. Despite partial results \citep{chudnovsky2007roots, wagner2010maxima}, the conjecture remains open, though \textcite{reynolds2026erdos} has verified it for all 447\,672\,596 trees on at most 26~vertices.

Why should biologists care about a graph-theoretic polynomial? In a protein interaction network whose local topology is tree-like \citep{milo2002network}, an independent set is a collection of proteins no two of which interact directly~-- a non-interacting module. In a dendritic arbor, it's a set of compartments that don't share a branch point. The shape of $\indpoly$~-- its mode, its width, its \term{near-miss ratio}~$\nm(T)$~-- characterises how the tree's topology distributes combinatorial freedom. The \term{hard-core model} from statistical physics \citep{galvin2004weighted, scott2005repulsive} assigns an occupation probability $\occ(v)$ to each vertex; these probabilities serve as centrality measures that reflect a vertex's role in the tree's independent-set structure.

This paper applies results from a companion study \citep{reynolds2026erdos} to real biological trees. We compute independence polynomials for neuronal reconstructions from NeuroMorpho.Org \citep{ascoli2007neuromorpho} and phylogenetic trees from the Open Tree of Life \citep{hinchliff2015synthesis}, confirming unimodality and log-concavity across all specimens. We interpret three results in biological terms: the Hub Exclusion Lemma (high-degree vertices are excluded from maximal independent configurations), the hard-core edge bound ($\occ(u) + \occ(v) < 2/3$ for adjacent vertices), and the near-miss ratio as a measure of how close a tree's topology sits to the combinatorial boundary.

\section{Mathematical background}

\subsection{Independence polynomials and unimodality}

A \term{graph} $G = (V, E)$ is a set of \term{vertices}~$V$ and a set of \term{edges}~$E$, each edge joining two vertices. A \term{tree} is a connected graph with no cycles. An \term{independent set} in~$G$ is a subset $S \subseteq V$ in which no two vertices share an edge. The \term{independence number} $\alpha(G)$ is the size of a largest independent set.

Write $i_k(G)$ for the number of independent sets of size~$k$, with $i_0 = 1$ (the empty set). The \term{independence polynomial} is
\[
  \indpoly \;=\; \sum_{k=0}^{\alpha(T)} i_k(T)\, x^k.
\]
This polynomial packages all the independent-set counts into a single object. Its coefficient sequence $i_0, i_1, \ldots, i_{\alpha}$ is \term{unimodal} if it rises to a single peak and then falls: there's an index~$p$ (the \term{mode}) with $i_0 \le i_1 \le \cdots \le i_p \ge i_{p+1} \ge \cdots \ge i_\alpha$.

\textcite{alavi1987vertex} conjectured that every tree's independence polynomial is unimodal. Despite decades of partial results~-- including special cases such as paths, caterpillars, and spiders~-- the conjecture remains open \citep{wagner2010maxima}. \textcite{reynolds2026erdos} verified it computationally for all 447\,672\,596 trees on at most 26~vertices.

A stronger property is \term{log-concavity}: $i_k^2 \ge i_{k-1}\, i_{k+1}$ for all~$k$. Log-concavity implies unimodality, but not conversely. \textcite{chudnovsky2007roots} proved that independence polynomials of claw-free graphs are real-rooted, which forces log-concavity for that class. Trees aren't generally claw-free, however, and at $n = 26$ two trees have log-concavity failures~-- yet both remain unimodal \citep{reynolds2026erdos}.

\subsection{The hard-core model}

The \term{hard-core model} at fugacity $\lambda = 1$ places a uniform distribution over all independent sets of~$T$. Each IS is equally likely. This model originates in statistical physics, where it describes non-overlapping particle occupation on a lattice; here the lattice is a tree \citep{galvin2004weighted, scott2005repulsive}.

Under this distribution, the \term{occupation probability} of a vertex~$v$ is $\occ(v) = \Pr[v \in S]$, the probability that~$v$ belongs to a uniformly random independent set. These probabilities serve as a kind of centrality measure: $\occ(v)$ reflects how strongly the tree's topology favours including~$v$ in an independent configuration.

The expected size of a random independent set is
\[
  \mu \;=\; \sum_{v \in V} \occ(v) \;=\; \frac{I'(T;\, 1)}{I(T;\, 1)},
\]
where $I'$ is the derivative. This connects the polynomial's analytic properties to a statistical quantity: $\mu$ is the mean of the coefficient distribution, weighted by the $i_k$.

The companion paper proves a structural constraint on adjacent occupation probabilities. For any edge $uv$ in a tree on $n \ge 3$ vertices,
\[
  \occ(u) + \occ(v) \;<\; \frac{2}{3}.
\]
This \term{edge bound} has an immediate consequence: the set $\{v : \occ(v) > 1/3\}$ is automatically an independent set, since two adjacent vertices can't both exceed~$1/3$. High-probability vertices repel each other. In biological terms, if two proteins interact directly, they can't both be highly likely members of a random non-interacting module.

\subsection{Structural reduction: the 1-Private framework}

The companion paper develops a structural reduction that isolates the combinatorially hardest trees. The key concept is that of a \term{private neighbour}. Given a maximal independent set~$S$, a non-$S$ vertex~$w$ is \term{private} to $u \in S$ if $u$ is the only $S$-member adjacent to~$w$. A maximal independent set is \term{1-Private} if every vertex in~$S$ has at most one private neighbour.

Write $d_{\mathrm{leaf}}(v)$ for the number of leaf-children of vertex~$v$. The \term{Hub Exclusion Lemma} constrains how high-degree vertices interact with 1-Private independent sets.

\medskip
\noindent\textbf{Hub Exclusion Lemma} \citep{reynolds2026erdos}.
\textit{If $d_{\mathrm{leaf}}(v) \ge 2$ and $S$ is a 1-Private maximal independent set, then $v \notin S$, and all leaf-children of~$v$ lie in~$S$.}
\medskip

The logic is direct: if~$v$ were in~$S$, each of its $\ge 2$ leaf-children would be private to~$v$, violating the 1-Private bound. Since $v \notin S$, domination forces every leaf-child into~$S$.

The \term{Transfer Lemma} ensures this reduction propagates. Pruning a hub~$v$ together with its leaf-children produces a residual tree (or forest)~$T'$, and the restriction $S' = S \cap V(T')$ remains a 1-Private maximal independent set in~$T'$. Iterating over all hubs with $d_{\mathrm{leaf}} \ge 2$ terminates at a residual in which every vertex has at most one leaf-child~-- a structurally simpler class of trees.

The reduction has practical implications for biological networks. Hub vertices in protein interaction networks~-- proteins with many single-degree interaction partners~-- are automatically excluded from 1-Private configurations while their partners are included. This parallels the empirical observation that hub proteins tend to be regulated rather than regulators.

Finally, the \term{near-miss ratio} $\nm(T)$ quantifies how close a tree sits to the unimodality boundary. Let $j_0$ be the first index where $i_{j_0} > i_{j_0+1}$ (the first strict descent in the coefficient sequence). Then
\[
  \nm(T) \;=\; \max_{j > j_0} \frac{i_{j+1}(T)}{i_j(T)}.
\]
A value exceeding~1 would be a unimodality violation; values near~1 indicate the tree's topology nearly permits one. The companion paper shows that for trees obtained by attaching $s$~pendant leaves to a fixed hub, $\nm(s) = 1 - C/s + O(1/s^2)$ with $C \in [4, 8)$. The margin shrinks as~$s$ grows but doesn't vanish.

\section{Biological trees as independence structures}

\subsection{Neuronal dendritic arbors}

TODO

\subsection{Phylogenetic trees}

TODO

\subsection{Transcription regulatory motifs}

TODO

\section{Computational results}

\subsection{Data and methods}

TODO

\subsection{Unimodality and log-concavity}

TODO

\subsection{Near-miss ratio and robustness}

TODO

\subsection{Hub exclusion in biological networks}

TODO

\section{Discussion}

\subsection{The independence polynomial as a network descriptor}

TODO

\subsection{Universality: physics, chemistry, biology}

TODO

\subsection{Limitations and open questions}

TODO

\section{Conclusion}

TODO: Write conclusion.

\section*{Acknowledgements}

% TODO: Replace with the actual models used on this project and their current versions.
% Check model names: frontier models change frequently (e.g., GPT-4 → GPT-4o → o3).
This paper was drafted with the assistance of large language models. All content has been reviewed and revised by the author, who takes full responsibility for the final text.

\newpage
\printbibliography

\end{document}
