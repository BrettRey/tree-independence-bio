\documentclass[12pt]{article}

% House style preamble
% !TEX TS-program = xelatex
% File: preamble.tex
% Purpose: Brett Reynolds house style LaTeX preamble
% Version: 2.0.0 (December 2025 typography redesign)
%
% Usage: % !TEX TS-program = xelatex
% File: preamble.tex
% Purpose: Brett Reynolds house style LaTeX preamble
% Version: 2.0.0 (December 2025 typography redesign)
%
% Usage: % !TEX TS-program = xelatex
% File: preamble.tex
% Purpose: Brett Reynolds house style LaTeX preamble
% Version: 2.0.0 (December 2025 typography redesign)
%
% Usage: \input{.house-style/preamble.tex} in main document
%
% Compiler: XeLaTeX (default) or LuaLaTeX
%           NOT pdfLaTeX - requires fontspec for fonts

% ===========================
% FONTS
% ===========================
\usepackage{fontspec}

\setmainfont{EB Garamond}[
  Numbers=OldStyle,
  Ligatures=TeX,
]

% IPA fallback font
\newfontfamily\ipafont{Charis SIL}
\newcommand{\ipa}[1]{{\ipafont #1}}

% Lining figures when needed (tables, years in isolation)
\providecommand{\liningnums}[1]{{\addfontfeatures{Numbers=Lining}#1}}

% Monospace
\setmonofont{Inconsolata}[Scale=MatchLowercase]

% ===========================
% PAGE LAYOUT
% ===========================
\usepackage[
  letterpaper,
  inner=1.25in,
  outer=1in,
  top=1in,
  bottom=1.25in,
  marginparwidth=0.6in,
]{geometry}

\usepackage[british]{babel}
\usepackage[final]{microtype}

% ===========================
% HEADINGS
% ===========================
\usepackage{titlesec}

% Section: small caps, number in margin
\titleformat{\section}
  {\normalfont\scshape}
  {\llap{\thesection\quad}}
  {0pt}
  {}

% Subsection: small caps sentence case
\titleformat{\subsection}
  {\normalfont\scshape}
  {\thesubsection\quad}
  {0pt}
  {}

% Subsubsection: upright
\titleformat{\subsubsection}
  {\normalfont}
  {\thesubsubsection\quad}
  {0pt}
  {}

% Spacing
\titlespacing*{\section}{0pt}{2ex plus 1ex minus .2ex}{1ex plus .2ex}
\titlespacing*{\subsection}{0pt}{1.5ex plus 1ex minus .2ex}{0.5ex plus .2ex}
\titlespacing*{\subsubsection}{0pt}{1ex plus 0.5ex minus .1ex}{0.3ex plus .1ex}

% ===========================
% RUNNING HEADS
% ===========================
\usepackage{fancyhdr}
\pagestyle{fancy}
\fancyhf{}
\fancyhead[LE]{\small\scshape\leftmark}
\fancyhead[RO]{\small\scshape\@title}
\fancyfoot[LE,RO]{\thepage}
\renewcommand{\headrulewidth}{0pt}

% For oneside documents (most papers):
\fancyhead[L]{\small\scshape\leftmark}
\fancyhead[R]{\small\thepage}
\fancyfoot{}

% ===========================
% COLORS & HYPERLINKS
% ===========================
\usepackage{xcolor}
\definecolor{linkmaroon}{RGB}{128, 0, 32}

\usepackage{hyperref}
\hypersetup{
  colorlinks=true,
  linkcolor=linkmaroon,
  citecolor=linkmaroon,
  urlcolor=linkmaroon,
  pdfauthor={Brett Reynolds},
}

\usepackage{orcidlink}

% ===========================
% QUOTATIONS
% ===========================
\usepackage[style=american]{csquotes} % Double quotes outer (Oxford spelling kept via babel)
\MakeOuterQuote{"}

% ===========================
% SEMANTIC MACROS
% ===========================
% Terms (linguistic concepts being introduced/defined)
\newcommand{\term}[1]{\textsc{#1}}

% Mentions (words/expressions under discussion)
\newcommand{\mention}[1]{\textit{#1}}

% Mentions in headings (angle brackets instead of italics)
% Robust: headings are moving arguments (written to .toc/.aux)
\DeclareRobustCommand{\mentionhead}[1]{⟨\textup{#1}⟩}

% Object language (cited forms, foreign words)
\newcommand{\olang}[1]{\textit{#1}}

% Small-caps abbreviations for glosses
\newcommand{\abbr}[1]{\textsc{#1}}

% Cross-linguistic subscript marker
\usepackage{marvosym}
\newcommand{\crossmark}{\textsubscript{\Cross}}

% ===========================
% MATHS AND SYMBOLS
% ===========================
\usepackage{amsmath,amssymb}

% ===========================
% LINGUISTIC EXAMPLES
% ===========================
\usepackage{langsci-gb4e}
\makeatletter
\@ifundefined{noautomath}{}{\noautomath}
\makeatother

% Judgement markers
\newcommand{\ungram}[1]{*\!#1}
\newcommand{\marg}[1]{?\!#1}
\newcommand{\odd}[1]{\#\!#1}

% ===========================
% LISTS
% ===========================
\usepackage{enumitem}
\setlist{nosep, leftmargin=*}
\setlist[enumerate]{label=\arabic*.}
\setlist[itemize]{label=--}

% ===========================
% TABLES & FIGURES
% ===========================
\usepackage{booktabs}
\usepackage{array}
\usepackage{graphicx}
\graphicspath{{figures/}}

% ===========================
% BIBLIOGRAPHY
% ===========================
\usepackage[backend=biber,style=apa,natbib=true,doi=true,isbn=false,url=true]{biblatex}
\addbibresource{references.bib}
\IfFileExists{references-local.bib}{\addbibresource{references-local.bib}}{}

% Possessive citation: Author's (Year)
\newcommand{\posscite}[1]{\citeauthor{#1}'s (\citeyear{#1})}

% ===========================
% UTILITIES
% ===========================
\usepackage{xspace}
\newcommand{\eg}{e.g.,\xspace}
\newcommand{\ie}{i.e.,\xspace}
\newcommand{\etc}{etc.\xspace}
 in main document
%
% Compiler: XeLaTeX (default) or LuaLaTeX
%           NOT pdfLaTeX - requires fontspec for fonts

% ===========================
% FONTS
% ===========================
\usepackage{fontspec}

\setmainfont{EB Garamond}[
  Numbers=OldStyle,
  Ligatures=TeX,
]

% IPA fallback font
\newfontfamily\ipafont{Charis SIL}
\newcommand{\ipa}[1]{{\ipafont #1}}

% Lining figures when needed (tables, years in isolation)
\providecommand{\liningnums}[1]{{\addfontfeatures{Numbers=Lining}#1}}

% Monospace
\setmonofont{Inconsolata}[Scale=MatchLowercase]

% ===========================
% PAGE LAYOUT
% ===========================
\usepackage[
  letterpaper,
  inner=1.25in,
  outer=1in,
  top=1in,
  bottom=1.25in,
  marginparwidth=0.6in,
]{geometry}

\usepackage[british]{babel}
\usepackage[final]{microtype}

% ===========================
% HEADINGS
% ===========================
\usepackage{titlesec}

% Section: small caps, number in margin
\titleformat{\section}
  {\normalfont\scshape}
  {\llap{\thesection\quad}}
  {0pt}
  {}

% Subsection: small caps sentence case
\titleformat{\subsection}
  {\normalfont\scshape}
  {\thesubsection\quad}
  {0pt}
  {}

% Subsubsection: upright
\titleformat{\subsubsection}
  {\normalfont}
  {\thesubsubsection\quad}
  {0pt}
  {}

% Spacing
\titlespacing*{\section}{0pt}{2ex plus 1ex minus .2ex}{1ex plus .2ex}
\titlespacing*{\subsection}{0pt}{1.5ex plus 1ex minus .2ex}{0.5ex plus .2ex}
\titlespacing*{\subsubsection}{0pt}{1ex plus 0.5ex minus .1ex}{0.3ex plus .1ex}

% ===========================
% RUNNING HEADS
% ===========================
\usepackage{fancyhdr}
\pagestyle{fancy}
\fancyhf{}
\fancyhead[LE]{\small\scshape\leftmark}
\fancyhead[RO]{\small\scshape\@title}
\fancyfoot[LE,RO]{\thepage}
\renewcommand{\headrulewidth}{0pt}

% For oneside documents (most papers):
\fancyhead[L]{\small\scshape\leftmark}
\fancyhead[R]{\small\thepage}
\fancyfoot{}

% ===========================
% COLORS & HYPERLINKS
% ===========================
\usepackage{xcolor}
\definecolor{linkmaroon}{RGB}{128, 0, 32}

\usepackage{hyperref}
\hypersetup{
  colorlinks=true,
  linkcolor=linkmaroon,
  citecolor=linkmaroon,
  urlcolor=linkmaroon,
  pdfauthor={Brett Reynolds},
}

\usepackage{orcidlink}

% ===========================
% QUOTATIONS
% ===========================
\usepackage[style=american]{csquotes} % Double quotes outer (Oxford spelling kept via babel)
\MakeOuterQuote{"}

% ===========================
% SEMANTIC MACROS
% ===========================
% Terms (linguistic concepts being introduced/defined)
\newcommand{\term}[1]{\textsc{#1}}

% Mentions (words/expressions under discussion)
\newcommand{\mention}[1]{\textit{#1}}

% Mentions in headings (angle brackets instead of italics)
% Robust: headings are moving arguments (written to .toc/.aux)
\DeclareRobustCommand{\mentionhead}[1]{⟨\textup{#1}⟩}

% Object language (cited forms, foreign words)
\newcommand{\olang}[1]{\textit{#1}}

% Small-caps abbreviations for glosses
\newcommand{\abbr}[1]{\textsc{#1}}

% Cross-linguistic subscript marker
\usepackage{marvosym}
\newcommand{\crossmark}{\textsubscript{\Cross}}

% ===========================
% MATHS AND SYMBOLS
% ===========================
\usepackage{amsmath,amssymb}

% ===========================
% LINGUISTIC EXAMPLES
% ===========================
\usepackage{langsci-gb4e}
\makeatletter
\@ifundefined{noautomath}{}{\noautomath}
\makeatother

% Judgement markers
\newcommand{\ungram}[1]{*\!#1}
\newcommand{\marg}[1]{?\!#1}
\newcommand{\odd}[1]{\#\!#1}

% ===========================
% LISTS
% ===========================
\usepackage{enumitem}
\setlist{nosep, leftmargin=*}
\setlist[enumerate]{label=\arabic*.}
\setlist[itemize]{label=--}

% ===========================
% TABLES & FIGURES
% ===========================
\usepackage{booktabs}
\usepackage{array}
\usepackage{graphicx}
\graphicspath{{figures/}}

% ===========================
% BIBLIOGRAPHY
% ===========================
\usepackage[backend=biber,style=apa,natbib=true,doi=true,isbn=false,url=true]{biblatex}
\addbibresource{references.bib}
\IfFileExists{references-local.bib}{\addbibresource{references-local.bib}}{}

% Possessive citation: Author's (Year)
\newcommand{\posscite}[1]{\citeauthor{#1}'s (\citeyear{#1})}

% ===========================
% UTILITIES
% ===========================
\usepackage{xspace}
\newcommand{\eg}{e.g.,\xspace}
\newcommand{\ie}{i.e.,\xspace}
\newcommand{\etc}{etc.\xspace}
 in main document
%
% Compiler: XeLaTeX (default) or LuaLaTeX
%           NOT pdfLaTeX - requires fontspec for fonts

% ===========================
% FONTS
% ===========================
\usepackage{fontspec}

\setmainfont{EB Garamond}[
  Numbers=OldStyle,
  Ligatures=TeX,
]

% IPA fallback font
\newfontfamily\ipafont{Charis SIL}
\newcommand{\ipa}[1]{{\ipafont #1}}

% Lining figures when needed (tables, years in isolation)
\providecommand{\liningnums}[1]{{\addfontfeatures{Numbers=Lining}#1}}

% Monospace
\setmonofont{Inconsolata}[Scale=MatchLowercase]

% ===========================
% PAGE LAYOUT
% ===========================
\usepackage[
  letterpaper,
  inner=1.25in,
  outer=1in,
  top=1in,
  bottom=1.25in,
  marginparwidth=0.6in,
]{geometry}

\usepackage[british]{babel}
\usepackage[final]{microtype}

% ===========================
% HEADINGS
% ===========================
\usepackage{titlesec}

% Section: small caps, number in margin
\titleformat{\section}
  {\normalfont\scshape}
  {\llap{\thesection\quad}}
  {0pt}
  {}

% Subsection: small caps sentence case
\titleformat{\subsection}
  {\normalfont\scshape}
  {\thesubsection\quad}
  {0pt}
  {}

% Subsubsection: upright
\titleformat{\subsubsection}
  {\normalfont}
  {\thesubsubsection\quad}
  {0pt}
  {}

% Spacing
\titlespacing*{\section}{0pt}{2ex plus 1ex minus .2ex}{1ex plus .2ex}
\titlespacing*{\subsection}{0pt}{1.5ex plus 1ex minus .2ex}{0.5ex plus .2ex}
\titlespacing*{\subsubsection}{0pt}{1ex plus 0.5ex minus .1ex}{0.3ex plus .1ex}

% ===========================
% RUNNING HEADS
% ===========================
\usepackage{fancyhdr}
\pagestyle{fancy}
\fancyhf{}
\fancyhead[LE]{\small\scshape\leftmark}
\fancyhead[RO]{\small\scshape\@title}
\fancyfoot[LE,RO]{\thepage}
\renewcommand{\headrulewidth}{0pt}

% For oneside documents (most papers):
\fancyhead[L]{\small\scshape\leftmark}
\fancyhead[R]{\small\thepage}
\fancyfoot{}

% ===========================
% COLORS & HYPERLINKS
% ===========================
\usepackage{xcolor}
\definecolor{linkmaroon}{RGB}{128, 0, 32}

\usepackage{hyperref}
\hypersetup{
  colorlinks=true,
  linkcolor=linkmaroon,
  citecolor=linkmaroon,
  urlcolor=linkmaroon,
  pdfauthor={Brett Reynolds},
}

\usepackage{orcidlink}

% ===========================
% QUOTATIONS
% ===========================
\usepackage[style=american]{csquotes} % Double quotes outer (Oxford spelling kept via babel)
\MakeOuterQuote{"}

% ===========================
% SEMANTIC MACROS
% ===========================
% Terms (linguistic concepts being introduced/defined)
\newcommand{\term}[1]{\textsc{#1}}

% Mentions (words/expressions under discussion)
\newcommand{\mention}[1]{\textit{#1}}

% Mentions in headings (angle brackets instead of italics)
% Robust: headings are moving arguments (written to .toc/.aux)
\DeclareRobustCommand{\mentionhead}[1]{⟨\textup{#1}⟩}

% Object language (cited forms, foreign words)
\newcommand{\olang}[1]{\textit{#1}}

% Small-caps abbreviations for glosses
\newcommand{\abbr}[1]{\textsc{#1}}

% Cross-linguistic subscript marker
\usepackage{marvosym}
\newcommand{\crossmark}{\textsubscript{\Cross}}

% ===========================
% MATHS AND SYMBOLS
% ===========================
\usepackage{amsmath,amssymb}

% ===========================
% LINGUISTIC EXAMPLES
% ===========================
\usepackage{langsci-gb4e}
\makeatletter
\@ifundefined{noautomath}{}{\noautomath}
\makeatother

% Judgement markers
\newcommand{\ungram}[1]{*\!#1}
\newcommand{\marg}[1]{?\!#1}
\newcommand{\odd}[1]{\#\!#1}

% ===========================
% LISTS
% ===========================
\usepackage{enumitem}
\setlist{nosep, leftmargin=*}
\setlist[enumerate]{label=\arabic*.}
\setlist[itemize]{label=--}

% ===========================
% TABLES & FIGURES
% ===========================
\usepackage{booktabs}
\usepackage{array}
\usepackage{graphicx}
\graphicspath{{figures/}}

% ===========================
% BIBLIOGRAPHY
% ===========================
\usepackage[backend=biber,style=apa,natbib=true,doi=true,isbn=false,url=true]{biblatex}
\addbibresource{references.bib}
\IfFileExists{references-local.bib}{\addbibresource{references-local.bib}}{}

% Possessive citation: Author's (Year)
\newcommand{\posscite}[1]{\citeauthor{#1}'s (\citeyear{#1})}

% ===========================
% UTILITIES
% ===========================
\usepackage{xspace}
\newcommand{\eg}{e.g.,\xspace}
\newcommand{\ie}{i.e.,\xspace}
\newcommand{\etc}{etc.\xspace}


% Project-specific macros
\newcommand{\indpoly}{I(T;\,x)}
\newcommand{\nm}{\mathrm{nm}}
\newcommand{\occ}{P}
\DeclareMathOperator{\mode}{mode}

% Update PDF metadata
\hypersetup{
  pdftitle={Tree Independence Polynomials and Biological Network Motifs}
}

\title{Tree Independence Polynomials and Biological Network Motifs}
\author{Brett Reynolds \orcidlink{0000-0003-0073-7195}%
\thanks{Contact: \href{mailto:brett.reynolds@humber.ca}{brett.reynolds@humber.ca}}\\
Humber Polytechnic \& University of Toronto}
\date{\today}

\begin{document}
\maketitle

\begin{abstract}
The independence polynomial of a graph counts independent sets by size, packaging these counts into a single generating function whose shape encodes the graph's combinatorial structure. Trees (graphs with no cycles) are ubiquitous in biology: dendritic arbors carry signals from synapse to soma, phylogenies record the branching history of species, and transcription regulatory motifs have tree-like acyclic backbones. I computed exact independence polynomials for 27 real biological trees (15 neuronal arbors from NeuroMorpho.Org and 12 phylogenies from the Open Tree of Life, ranging from 46 to 30,310 vertices). All 27 are unimodal and log-concave, extending the computational verification of a long-standing conjecture to biological data for the first time. Three structural results translate cleanly into biological terms: high-degree vertices (soma, polytomies, transcription factors) are excluded from certain maximal independent configurations while peripheral elements (terminal dendrites, tip species, target genes) are included; adjacent vertices are constrained by the hard-core edge bound; and the near-miss ratio quantifies how close each tree's topology sits to the combinatorial boundary. A null-model comparison shows this convergence is a generic property of tree size; the per-vertex occupation probability, which captures hub-periphery asymmetry in regulatory networks, is where the framework adds the most biological value. The same polynomial appears as a partition function in statistical physics and a molecular descriptor in chemical graph theory, suggesting that the independence polynomial is a natural, discipline-spanning network descriptor, one whose global shape is generic but whose vertex-level decomposition may identify which nodes matter and which are structurally expendable.
\end{abstract}

\section{Introduction}

Nature has an inordinate fondness for trees. Phylogenies record the branching history of species \citep{hinchliff2015synthesis}. Dendritic arbors carry signals from synapse to soma. The backbone of a transcription regulatory network, stripped of its cycles, is a tree \citep{alon2007network}. In each case the branching pattern isn't just a shape; it constrains what the system can do. This paper asks what a single polynomial can reveal about that constraint — and why the same object keeps turning up in places that don't talk to each other.

The \term{independence polynomial} of a tree~$T$ counts the ways to select non-adjacent vertices. Call a set of vertices \term{independent} if no two are neighbours, and let $i_k(T)$ be the number of independent sets of size~$k$. The polynomial $\indpoly = \sum_k i_k(T)\, x^k$ packages these counts. A small example makes the definition concrete. Take a path of three vertices, $a$--$b$--$c$. Its independent sets are $\varnothing$, $\{a\}$, $\{b\}$, $\{c\}$, and $\{a,c\}$~-- five in all. Grouping by size gives $i_0 = 1$, $i_1 = 3$, $i_2 = 1$, so the polynomial is $1 + 3x + x^2$. The sequence 1,~3,~1 rises to a peak and then falls.

\textcite{alavi1987vertex} conjectured that this single-peak pattern holds for every tree: the coefficient sequence is always \term{unimodal}. This is Erd\H{o}s Problem~993. Despite partial results \citep{chudnovsky2007roots, levit2005independence}, the conjecture remains open, though \textcite{reynolds2026erdos} has verified it for all 1\,198\,738\,056 trees on at most 27~vertices.

Why should biologists care about a graph-theoretic polynomial? In a network whose local topology is tree-like, an independent set is a subset of nodes with no direct connections between them, a non-interacting subset (in the combinatorial, not the systems-biology, sense of \enquote{module}). In a dendritic arbor, it's a set of compartments that don't share a branch point. The shape of $\indpoly$ (its mode, its width, its \term{near-miss ratio}~$\nm(T)$) characterizes how the tree's topology distributes independent-set structure. The \term{hard-core model} from statistical physics \citep{galvin2004weighted, scott2005repulsive} assigns an occupation probability $\occ(v)$ to each vertex; these probabilities serve as centrality measures that reflect a vertex's role in the tree's independent-set structure.

This paper is a bridge: the mathematical results come from a companion study \citep{reynolds2026erdos}; the contribution here is to test them on real biological data and ask which ones carry biological meaning. I compute independence polynomials for neuronal reconstructions from NeuroMorpho.Org \citep{ascoli2007neuromorpho} and phylogenetic trees from the Open Tree of Life \citep{hinchliff2015synthesis}, confirming unimodality and log-concavity across all specimens. I interpret three results in biological terms: the Hub Exclusion Lemma (high-degree vertices are excluded from certain maximal independent configurations), the hard-core edge bound ($\occ(u) + \occ(v) < 2/3$ for adjacent vertices), and the near-miss ratio as a measure of how close a tree's topology sits to the combinatorial boundary.

\section{Independent sets in biological trees}

\subsection{Neuronal dendritic arbors}

Dendritic trees are literal trees. The standard SWC file format used by NeuroMorpho.Org \citep{ascoli2007neuromorpho} encodes each neuronal reconstruction as a rooted tree: every compartment (a short cylinder of membrane) is a vertex, and edges record the parent-child relationship along the dendrite. The graph has no cycles by construction (the format records only the branching skeleton, omitting gap junctions and other lateral connections).

The branching pattern of a dendrite constrains which compartments can act independently. An independent set in a dendritic tree is a set of compartments with no shared branch segments, non-adjacent functional units. The Hub Exclusion Lemma (Section~\ref{sec:1private}) makes a structural prediction here. Consider a branch point~$b$ with three terminal dendrites $t_1$, $t_2$, $t_3$. In every 1-Private maximal independent set (Section~\ref{sec:1private}), $b$ is forced out and all three terminals are forced in~-- the periphery dominates.

This matches biological intuition. Terminal arbors are the \enquote{active} sites (they receive synaptic input and initiate local dendritic computation), whereas branch points serve a structural, routing role.

The hard-core edge bound (Section~\ref{sec:hardcore}) reinforces the picture. For any parent-child pair along the dendrite, $\occ(\text{parent}) + \occ(\text{child}) < 2/3$, where $\occ(v)$ is the fraction of all independent sets containing vertex~$v$. Adjacent compartments can't both have high occupation probability.

Return to the path $a$--$b$--$c$. Vertex~$a$ appears in 2 of 5 independent sets, so $\occ(a) = 0.4$; the central vertex~$b$ appears in only 1, so $\occ(b) = 0.2$. The sum $\occ(a) + \occ(b) = 0.6 < 2/3$: the bound holds, and the peripheral vertex wins. Branching architecture concentrates independent-set membership at the periphery.

\subsection{Phylogenetic trees}

Phylogenies are trees by definition. Tips represent extant species; internal nodes represent ancestral divergence events. The Open Tree of Life \citep{hinchliff2015synthesis} assembles published phylogenies into a single supertree spanning over two million tips, a dataset that is both enormous and, at every local neighbourhood, a tree.

The independence polynomial $\indpoly$ encodes the structure of evolutionary branching. Each coefficient $i_k$ counts the number of ways to select $k$ non-adjacent nodes in the phylogeny. Its mode, width, and near-miss ratio characterize properties of the branching pattern that aren't captured by standard measures like balance or gamma statistics.

\term{Polytomies} (nodes where an ancestor splits into three or more descendants) are the hubs where Hub Exclusion applies. Most polytomies in published phylogenies are \term{soft} (artefacts of insufficient resolution rather than genuine simultaneous speciation), so Hub Exclusion here is a statement about the reconstruction's topology, not necessarily about the underlying biology.

A polytomy with $d_{\mathrm{leaf}} \ge 2$ descendant tips can't belong to any 1-Private maximal independent set (Section~\ref{sec:1private}); its tip descendants must. Tip species dominate the independent-set structure, and deep ancestors with many descendants don't participate.

The near-miss ratio $\nm(T)$ (Section~\ref{sec:nearmiss}) characterizes how \enquote{robust} the phylogeny's combinatorial structure is: values near~1 indicate the branching pattern sits close to the unimodality boundary.

\subsection{Transcription regulatory motifs}

\term{Single-Input Modules} (SIMs) are among the most common motifs in transcription regulatory networks: one transcription factor (TF) regulates $N$~target genes \citep{alon2007network, milo2002network}. Topologically, a SIM is a star, a tree with one hub and $N$~leaves. This makes it the simplest non-trivial case for the independence polynomial framework.

Hub Exclusion applies directly. The TF is the hub vertex with $d_{\mathrm{leaf}} = N \ge 2$ leaf-children, so it can't belong to any 1-Private maximal independent set (Section~\ref{sec:1private}); every target must. This mirrors the biological logic of a SIM: the target genes can be independently active (expressed or not), but the regulator's state constrains the whole module. The regulator isn't \enquote{free} in the way the targets are; it's the price of centrality.

The hard-core edge bound $\occ(\mathrm{TF}) + \occ(\mathrm{target}_i) < 2/3$ constrains co-occurrence between the regulator and any one of its targets. In the hard-core model, the TF has low occupation probability (it belongs to few independent sets relative to the total) while each target has high occupation probability. For a star $K_{1,N}$, the TF's occupation probability is $1/(2^N + 1)$, which drops toward zero as $N$ grows: a TF with 5~targets has occupation probability just $1/33 \approx 0.03$. The targets collectively dominate the independent-set structure, exactly the asymmetry that makes SIMs function as regulatory amplifiers.

The same asymmetry recurs across all three systems: the branch point is out and its terminals are in; the polytomy is out and its tip species are in; the transcription factor is out and its targets are in.

\section{Computational results}

\subsection{Data and methods}

I analysed 27 biological trees drawn from two sources. The sample is illustrative, not exhaustive; a systematic survey of NeuroMorpho.Org's 260\,000+ reconstructions is computationally feasible and, in the usual manner, left for future work.

From NeuroMorpho.Org \citep{ascoli2007neuromorpho} I selected 15 neuronal reconstructions in SWC format, choosing trees that span a range of sizes, species (Drosophila, macaque, human, mouse, rat), and cell types (pyramidal cells, Purkinje cells, interneurons, principal cells). Tree sizes range from $n = 113$ (Drosophila ddaC) to $n = 17{,}992$ (human pyramidal).

From the Open Tree of Life \citep{hinchliff2015synthesis} I extracted 12 phylogenetic subtrees by querying named clades via the OTL API and exporting induced subtrees in Newick format. Zero-length branches were collapsed, producing the polytomies discussed in Section~2.2. Tree sizes range from $n = 46$ (Homininae) to $n = 30{,}310$ (Aves).

Independence polynomials were computed exactly using arbitrary-precision integer arithmetic via \texttt{indpoly.py}, the algorithm described in the companion paper \citep{reynolds2026erdos}. Format conversion from SWC and Newick files used \texttt{bio\_trees.py}. Runtime was sub-second for trees with $n < 1{,}000$ and approximately 142~minutes for the largest tree ($n = 30{,}310$). Table~\ref{tab:results} summarizes the results for all 27~trees.

\begin{table}[htbp]
\centering
\caption{Independence polynomial properties of biological trees. $n$~= number of nodes, $\alpha$~= independence number, mode~= index of largest coefficient, $\mathrm{nm}$~= near-miss ratio.}
\label{tab:results}
\small
\begin{tabular}{@{}llrrrr@{}}
\toprule
Tree & Species & $n$ & $\alpha$ & mode & $\mathrm{nm}$ \\
\midrule
\multicolumn{6}{@{}l}{\textit{Neuronal arbors (NeuroMorpho.org)}} \\
\addlinespace[2pt]
  Drosophila ddaC (A5) & Drosophila & 113 & 58 & 32 & 0.8478 \\
  Drosophila ddaC (A6) & Drosophila & 166 & 83 & 46 & 0.9317 \\
  Monkey L3 pyramidal (041) & Macaque & 1,038 & 525 & 290 & 0.9813 \\
  Monkey L3 pyramidal (001) & Macaque & 1,274 & 646 & 356 & 0.9873 \\
  Human aspiny interneuron & Human & 1,498 & 752 & 416 & 0.9865 \\
  Mouse Purkinje (P35) & Mouse & 1,716 & 891 & 491 & 0.9896 \\
  Rat interneuron & Rat & 1,941 & 978 & 540 & 0.9927 \\
  Mouse principal cell & Mouse & 2,212 & 1,108 & 612 & 0.9944 \\
  Rat interneuron (IDC) & Rat & 2,774 & 1,392 & 770 & 0.9930 \\
  Rat pyramidal (n419) & Rat & 4,229 & 2,122 & 1,174 & 0.9958 \\
  Mouse Purkinje (P43) & Mouse & 4,156 & 2,126 & 1,172 & 0.9954 \\
  Rat basket interneuron & Rat & 6,378 & 3,199 & 1,767 & 0.9970 \\
  Rat interneuron (A3) & Rat & 7,649 & 3,843 & 2,123 & 0.9981 \\
  Human pyramidal (06) & Human & 10,451 & 5,236 & 2,895 & 0.9980 \\
  Human pyramidal (05) & Human & 17,992 & 9,011 & 4,981 & 0.9990 \\
\addlinespace[6pt]
\multicolumn{6}{@{}l}{\textit{Phylogenetic trees (Open Tree of Life)}} \\
\addlinespace[2pt]
  Homininae & --- & 46 & 28 & 15 & 0.7700 \\
  Delphinidae & --- & 107 & 71 & 36 & 0.9184 \\
  Felidae & --- & 144 & 100 & 52 & 0.9479 \\
  Salamandridae & --- & 157 & 105 & 54 & 0.9423 \\
  Mustelidae & --- & 188 & 122 & 64 & 0.9488 \\
  Canidae & --- & 367 & 241 & 125 & 0.9812 \\
  Equidae & --- & 372 & 223 & 117 & 0.9612 \\
  Accipitridae & --- & 884 & 601 & 311 & 0.9857 \\
  Papilionidae & --- & 1,170 & 929 & 472 & 0.9917 \\
  Primates & --- & 1,333 & 872 & 452 & 0.9941 \\
  Cetacea & --- & 1,387 & 971 & 498 & 0.9911 \\
  Aves & --- & 30,310 & 20,357 & 10,532 & 0.9997 \\
\bottomrule
\end{tabular}
\end{table}

\subsection{Unimodality and log-concavity}

All 27 biological trees have unimodal and log-concave independence polynomials. This extends the exhaustive verification of all abstract trees on $\le 27$ vertices \citep{reynolds2026erdos} to real biological trees of much larger size, up to $n = 30{,}310$ vertices. Mode positions cluster near $\alpha/2$ (mode$/\alpha \approx 0.55$ for most trees), consistent with the symmetric bell-curve shape visible in Figure~\ref{fig:polyshapes}. Neither the neuronal arbors nor the phylogenetic trees produced any counterexample to the unimodality conjecture.

\subsection{Near-miss ratio and robustness}

The \term{near-miss ratio} $\nm(T)$ ranges from 0.770 (Homininae, $n = 46$) to 0.9997 (Aves, $n = 30{,}310$). A least-squares fit gives $\nm = 1 - 10.8/n$ ($R^2 = 0.934$, Figure~\ref{fig:nmvsn}), consistent with the asymptotic $1 - C/s$ derived in the companion paper \citep{reynolds2026erdos}. (The fitted constant is larger than the theoretical $C \in [4, 8)$ because the asymptotic uses the pendant count~$s$, not the total vertex count~$n$.)

The pattern admits a clean interpretation. Larger biological trees sit closer to the combinatorial boundary (their coefficient sequences nearly fail to be unimodal) but they never cross it. I call this the \term{dilution effect}: as $n$~grows, the polynomial acquires more coefficients and the sequence smooths, making near-misses more likely but actual violations no closer. Small trees ($n < 100$) have $\nm$ well below~0.9; large trees ($n > 1{,}000$) all exceed~0.98.

The two data types occupy different regions of Figure~\ref{fig:nmvsn}. Neuronal arbors, with their long chains of degree-2 vertices, tend toward higher~$\nm$ at a given~$n$ than phylogenies of comparable size. Phylogenies with many polytomies have more combinatorial structure per vertex, which keeps $\nm$ slightly lower. Both populations follow the same asymptotic envelope.

The correlation is strong and statistically robust. Spearman's $\rho = 0.973$ (95\% CI $[0.899,\, 0.993]$, $p < 0.001$, $n = 27$) for the pooled sample; within neurons alone $\rho = 0.989$ ($p < 0.001$, $n = 15$) and within phylogenies $\rho = 0.965$ ($p < 0.001$, $n = 12$).

The near-miss ratio also correlates with standard tree-shape statistics: Sackin index \citep{sackin1972good} ($\rho = 0.924$, $p < 0.001$) and Colless-like index \citep{colless1982review} ($\rho = 0.937$, $p < 0.001$), but not with leaf count ($\rho = 0.361$, $p = 0.064$). The non-correlation with leaf count is consistent with the hub-pendant asymptotic $\nm(s) = 1 - C/s$ derived in the companion paper, where $s$ is the total pendant count, not the leaf count: $\nm$ tracks tree size and branching structure, not the number of tips specifically.

\begin{figure}[htbp]
  \centering
  \includegraphics[width=\textwidth]{figures/nm_regression.pdf}
  \caption{Near-miss ratio $\nm(T)$ versus tree size~$n$ for 27~biological trees. \textit{Top:} least-squares fit $\nm = 1 - 10.8/n$ ($R^2 = 0.934$). Blue circles: neuronal arbors; red triangles: phylogenies. \textit{Bottom:} residuals from the fit, labelled by tree. Phylogenies with heavy polytomies (Equidae, Salamandridae) tend to fall below the curve; long-chain neuronal arbors tend to sit above it.}
  \label{fig:nmvsn}
\end{figure}

\begin{figure}[htbp]
  \centering
  \includegraphics[width=\textwidth]{figures/poly_shapes.pdf}
  \caption{Normalized independence polynomial coefficients $i_k / \max(i_k)$ versus $k/\alpha$ for four representative trees. The universal unimodal bell-curve shape is evident across both data types and a wide range of tree sizes.}
  \label{fig:polyshapes}
\end{figure}

\subsection{Hub exclusion in biological networks}

The Hub Exclusion Lemma applies wherever a vertex has two or more leaf-children. The two data types differ in where this condition holds. Phylogenies have many hub vertices (polytomies where an ancestor diverges into three or more descendants simultaneously), so Hub Exclusion is active throughout the tree. Neuronal arbors, by contrast, consist mostly of degree-2 chains (unbranched dendritic segments) with occasional branch points, and hub exclusion applies only at those branch points.

In phylogenies, unresolved radiations (polytomies with $d_{\mathrm{leaf}} \ge 2$) can't appear in any 1-Private maximal independent configuration; their tip species must. The asymmetry is stronger in phylogenies than in neuronal arbors because phylogenies have higher maximum vertex degree. In both systems, peripheral elements (terminal dendrites, extant species) carry more weight in the independent-set structure than central hubs. The independence polynomial formalizes this intuition as a theorem.

\subsection{Occupation probabilities}

The Hub Exclusion Lemma is a qualitative statement (hubs are excluded from certain configurations). The occupation probability $\occ(v)$ makes this quantitative: it measures the fraction of all independent sets that include vertex~$v$. I computed $\occ(v)$ for every vertex in each of the 27~biological trees using a two-pass $O(n)$ dynamic program on the rooted tree (bottom-up subtree counts, then top-down rest-of-tree counts; see Section~\ref{sec:hardcore}).

Figure~\ref{fig:pvdeg} plots $\occ(v)$ against vertex degree for three representative trees. The pattern is consistent across all 27~trees: high-degree vertices cluster at low $\occ(v)$, and leaves ($\deg = 1$) cluster near $\occ(v) \approx 0.5$. No vertex with degree $\ge 3$ exceeds the $1/3$ threshold imposed by the edge bound (Section~\ref{sec:hardcore}), confirming that high-degree vertices are not just excluded from 1-Private configurations but are underrepresented across all independent sets.

For the Homininae phylogeny ($n = 46$), the highest-degree vertex (degree~6) has $\occ(v) = 0.023$, while leaves average $\occ(v) = 0.489$, a 21-fold difference. In the Primates phylogeny ($n = 1{,}333$), the most connected vertex has $\occ(v) < 0.01$. The pattern holds for neuronal arbors too, though the degree range is narrower (most vertices have degree~2).

\begin{figure}[htbp]
  \centering
  \includegraphics[width=\textwidth]{figures/pv_vs_degree.pdf}
  \caption{Occupation probability $\occ(v)$ versus vertex degree for three representative trees. The dotted line marks $\occ(v) = 1/3$, the edge-bound threshold above which two adjacent vertices cannot both lie. Leaves cluster near $\occ(v) \approx 0.5$; high-degree vertices are confined to low $\occ(v)$.}
  \label{fig:pvdeg}
\end{figure}

\subsection{Null-model comparison}

Does the near-miss ratio of biological trees differ from what random trees of the same size would produce? To test this, I generated random labelled trees (uniform via Prüfer sequence) matched to each biological tree's vertex count and computed their independence polynomials. Figure~\ref{fig:nullnm} shows the null distributions alongside the observed biological values.

The answer is mixed. For most tree sizes, the biological $\nm$ falls within the interquartile range of the null distribution. The near-miss ratio is driven primarily by tree size, not by the specific branching topology: a random tree with $n = 1{,}000$ vertices has $\nm \approx 0.987$, close to the biological values at that size. This result strengthens the $1 - C/n$ asymptotic (Section~\ref{sec:nearmiss}): the convergence toward~1 is a generic property of large trees, not a consequence of biological architecture.

A few phylogenies with heavy polytomies (Homininae, Delphinidae) fall slightly below the null median, consistent with the observation in Section~3.3 that polytomy-rich phylogenies have lower~$\nm$ at a given~$n$. But the deviations are modest, and no biological tree is an outlier relative to its null distribution.

The Prüfer-sequence null is conservative: it samples uniformly over all labelled trees, including topologies (long paths, high-degree stars) that no biological process would produce. A null model matched to a biologically plausible generating process (Yule branching \citep{yule1925mathematical} for phylogenies, or constrained-degree models for dendrites) might reveal deviations that the uniform null obscures.

\begin{figure}[htbp]
  \centering
  \includegraphics[width=\textwidth]{figures/null_model_nm.pdf}
  \caption{Near-miss ratio for biological trees (coloured points) versus null distributions from uniform random trees of the same size (box plots). Blue circles: neuronal arbors; red triangles: phylogenies. Most biological values fall within the null interquartile range, indicating that $\nm$ is driven primarily by tree size.}
  \label{fig:nullnm}
\end{figure}

\subsection{Concrete example: \textit{E.\@ coli} regulatory SIMs}

Single-Input Modules (SIMs) in the \textit{E.\@ coli} transcription regulatory network provide a concrete test of the framework on known biological motifs. Each SIM is a star tree $K_{1,N}$ with a transcription factor (TF) as hub and $N$~target operons as leaves. Using published SIM fan-outs from \textcite{alon2007network}, I computed independence polynomials for 13~well-characterized SIMs (Table~\ref{tab:sims}).

All 13 are unimodal and log-concave. The largest, CRP ($N = 73$ targets), has $\nm = 0.897$, high enough to confirm that even moderate-size stars sit well within the unimodal regime. The occupation probabilities confirm Hub Exclusion quantitatively: $\occ(\mathrm{CRP}) = 1/(2^{73} + 1) \approx 10^{-22}$, while $\occ(\mathrm{target}) = 2^{72}/(2^{73} + 1) \approx 0.500$. The TF is vanishingly unlikely to appear in a random independent set; its targets dominate.

\begin{table}[htbp]
\centering
\caption{Independence polynomial properties of \textit{E.\@ coli} SIMs. $N$~= number of target operons, $n = N + 1$~= total vertices. $\occ(\mathrm{TF}) = 1/(2^N + 1)$. Fan-out data from \textcite{alon2007network}.}
\label{tab:sims}
\small
\begin{tabular}{@{}lrrrrl@{}}
\toprule
TF & $N$ & $\alpha$ & mode & $\nm$ & $\log_{10}\occ(\mathrm{TF})$ \\
\midrule
CRP & 73 & 73 & 36 & 0.8974 & $-22.0$ \\
FNR & 27 & 27 & 13 & 0.7500 & $-8.1$ \\
IHF & 18 & 18 & 9 & 0.7273 & $-5.4$ \\
LexA & 16 & 16 & 8 & 0.7000 & $-4.8$ \\
Fis & 16 & 16 & 8 & 0.7000 & $-4.8$ \\
ArcA & 15 & 15 & 7 & 0.6000 & $-4.5$ \\
Lrp & 13 & 13 & 6 & 0.5556 & $-3.9$ \\
NarL & 11 & 11 & 5 & 0.5000 & $-3.3$ \\
PurR & 10 & 10 & 5 & 0.5714 & $-3.0$ \\
ArgR & 9 & 9 & 4 & 0.4286 & $-2.7$ \\
FlhDC & 8 & 8 & 4 & 0.5000 & $-2.4$ \\
MetJ & 7 & 7 & 3 & 0.3333 & $-2.1$ \\
TrpR & 4 & 4 & 2 & 0.2500 & $-1.2$ \\
\bottomrule
\end{tabular}
\end{table}

\section{Mathematical framework}

This section collects the formal definitions and results used informally in Sections~2 and~3. Readers who prefer definitions first may want to start here; those comfortable with the intuitive descriptions above can skip ahead to the Discussion and refer back as needed.

\subsection{Independence polynomials and unimodality}

A \term{graph} $G = (V, E)$ is a set of \term{vertices}~$V$ and a set of \term{edges}~$E$, each edge joining two vertices. A \term{tree} is a connected graph with no cycles. An \term{independent set} in~$G$ is a subset $S \subseteq V$ in which no two vertices share an edge (in a biological tree, a collection of nodes no two of which share a branch). The \term{independence number} $\alpha(G)$ is the size of a largest independent set.

Write $i_k(G)$ for the number of independent sets of size~$k$, with $i_0 = 1$ (the empty set). The \term{independence polynomial} is
\[
  \indpoly \;=\; \sum_{k=0}^{\alpha(T)} i_k(T)\, x^k.
\]
This polynomial packages all the independent-set counts into a single object. For the path $a$--$b$--$c$ from Section~1, $i_0 = 1$, $i_1 = 3$, $i_2 = 1$, so $I(P_3;\, x) = 1 + 3x + x^2$, with $\alpha = 2$.

The coefficient sequence $i_0, i_1, \ldots, i_{\alpha}$ is \term{unimodal} if it rises to a single peak and then falls: there's an index~$p$ (the \term{mode}) with $i_0 \le i_1 \le \cdots \le i_p \ge i_{p+1} \ge \cdots \ge i_\alpha$. For $P_3$ the sequence 1,~3,~1 peaks at $p = 1$: unimodal.

\textcite{alavi1987vertex} conjectured that every tree's independence polynomial is unimodal. Despite decades of partial results (including special cases such as paths, caterpillars, and spiders), the conjecture remains open \citep{levit2005independence}. \textcite{reynolds2026erdos} verified it computationally for all 1\,198\,738\,056 trees on at most 27~vertices.

A stronger property is \term{log-concavity}: $i_k^2 \ge i_{k-1}\, i_{k+1}$ for all~$k$. For $P_3$: $3^2 = 9 \ge 1 \cdot 1 = 1$. Log-concavity implies unimodality, but not conversely. \textcite{chudnovsky2007roots} proved that independence polynomials of claw-free graphs are real-rooted, which forces log-concavity for that class. But trees aren't generally claw-free, and at $n = 26$ two trees have log-concavity failures, but both remain unimodal \citep{reynolds2026erdos}.

\subsection{The hard-core model}\label{sec:hardcore}

The \term{hard-core model} from statistical physics describes non-overlapping particle occupation on a lattice, controlled by a parameter~$\lambda$ (the \term{fugacity}) that tunes how strongly the model favours large configurations \citep{galvin2004weighted, scott2005repulsive}. Here the lattice is a tree. At $\lambda = 1$ the distribution over independent sets is uniform: every independent set of~$T$ is equally likely.

Under this distribution, the \term{occupation probability} of a vertex~$v$ is $\occ(v) = \Pr[v \in S]$, the probability that~$v$ belongs to a uniformly random independent set. For $P_3 = a$--$b$--$c$, the five independent sets are $\varnothing$, $\{a\}$, $\{b\}$, $\{c\}$, $\{a,c\}$. Vertex~$a$ appears in two of them, so $\occ(a) = 2/5 = 0.4$. The central vertex~$b$ appears in only one, so $\occ(b) = 1/5 = 0.2$. Already a pattern: peripheral vertices have higher occupation probability than central ones.

The companion paper proves a structural constraint on adjacent occupation probabilities. For any edge $uv$ in a tree on $n \ge 3$ vertices,
\[
  \occ(u) + \occ(v) \;<\; \frac{2}{3}.
\]
For $P_3$, the edge $a$--$b$ gives $\occ(a) + \occ(b) = 0.4 + 0.2 = 0.6 < 2/3$. This \term{edge bound} has an immediate consequence: the set $\{v : \occ(v) > 1/3\}$ is automatically an independent set, since two adjacent vertices can't both exceed~$1/3$. High-probability vertices repel each other: popularity, here, is locally incompatible.\footnote{The edge bound formalizes a constraint on co-occurrence: adjacent vertices can't both have high occupation probability. This is structurally analogous to the homeostatic mechanisms that enforce property clustering in natural-kinds theory, but operating in reverse: the tree topology enforces exclusion rather than co-occurrence.} In biological terms, if two proteins interact directly, they can't both be highly likely members of a random non-interacting module.

\subsection{Structural reduction: the 1-Private framework}\label{sec:1private}

The companion paper develops a structural reduction that isolates the combinatorially hardest trees. The key concept is that of a \term{private neighbour}. Given a maximal independent set~$S$, a non-$S$ vertex~$w$ is \term{private} to $u \in S$ if $u$ is the only $S$-member adjacent to~$w$. A maximal independent set is \term{1-Private} if every vertex in~$S$ has at most one private neighbour.

In $P_3$, the two maximal independent sets are $\{b\}$ and $\{a,c\}$. In $\{b\}$, both $a$ and~$c$ are private to~$b$ (each has $b$ as its only $S$-neighbour), so $b$ has two private neighbours; not 1-Private. In $\{a,c\}$, the only non-$S$ vertex is~$b$, which is adjacent to both $a$ and~$c$, so $b$ isn't private to either one. Both $a$ and~$c$ have zero private neighbours: 1-Private. The sole 1-Private maximal independent set is $\{a,c\}$, which excludes the central vertex.

Write $d_{\mathrm{leaf}}(v)$ for the number of leaf-children of vertex~$v$. The \term{Hub Exclusion Lemma} constrains how high-degree vertices interact with 1-Private independent sets.

\medskip
\noindent\textbf{Hub Exclusion Lemma} \citep{reynolds2026erdos}.
\textit{If $d_{\mathrm{leaf}}(v) \ge 2$ and $S$ is a 1-Private maximal independent set, then $v \notin S$, and all leaf-children of~$v$ lie in~$S$.}
\medskip

The logic is direct: if~$v$ were in~$S$, each of its $\ge 2$ leaf-children would be private to~$v$, violating the 1-Private bound. Since $v \notin S$, domination forces every leaf-child into~$S$.

The \term{Transfer Lemma} ensures this reduction propagates. Pruning a hub~$v$ together with its leaf-children produces a residual tree (or forest)~$T'$, and the restriction $S' = S \cap V(T')$ remains a 1-Private maximal independent set in~$T'$. Iterating over all hubs with $d_{\mathrm{leaf}} \ge 2$ terminates at a residual in which every vertex has at most one leaf-child, a structurally simpler class of trees.

The reduction has practical implications for biological networks. Hub vertices (proteins with many single-degree interaction partners, transcription factors with many targets) can't survive in 1-Private configurations; their partners fill the gap.

\subsection{The near-miss ratio}\label{sec:nearmiss}

The \term{near-miss ratio} $\nm(T)$ quantifies how close a tree sits to the unimodality boundary. Let $j_0$ be the first index where $i_{j_0} > i_{j_0+1}$ (the first strict descent in the coefficient sequence). Then
\[
  \nm(T) \;=\; \max_{j > j_0} \frac{i_{j+1}(T)}{i_j(T)}.
\]
A value exceeding~1 would be a unimodality violation; values near~1 indicate the tree's topology nearly permits one. For $P_3$ with sequence 1,~3,~1, the first descent is at $j_0 = 1$ ($i_1 = 3 > i_2 = 1$), and there's no later ratio to compute, so $\nm$ is undefined (the sequence is too short to nearly fail). For the Aves phylogeny ($n = 30{,}310$), $\nm = 0.9997$: the coefficient sequence almost ticks back up after its peak, missing a violation by a hair.

The companion paper shows that for trees obtained by attaching $s$~pendant leaves to a fixed hub, $\nm(s) = 1 - C/s + O(1/s^2)$ with $C \in [4, 8)$. The margin shrinks as~$s$ grows but doesn't vanish.

\section{Discussion}

\subsection{The independence polynomial as a network descriptor}

Standard network descriptors (degree distribution, clustering coefficient, betweenness centrality, and domain-specific measures like Sholl analysis \citep{sholl1953dendritic} for dendrites or balance indices \citep{sackin1972good, colless1982review} for phylogenies) reduce a graph to a handful of scalars. The independence polynomial retains more: its full coefficient sequence $i_0, i_1, \ldots, i_\alpha$ encodes how combinatorial freedom is distributed across every possible selection size.

Mode position and near-miss ratio each capture a different facet of the tree's branching topology, and per-vertex occupation probabilities $\occ(v)$ add a third (Section~3.5). The near-miss ratio correlates strongly with standard tree-shape statistics (Sackin $\rho = 0.924$, Colless-like $\rho = 0.937$), which initially looks promising, until the null-model comparison (Section~3.6) shows that $\nm$ tracks tree size, not branching pattern. Biological trees are indistinguishable from random trees of the same size on this measure.

The correlation is real, but it isn't biological. This null result suggests a general criterion: a descriptor \term{accommodates} biological structure if its values on real specimens are distinguishable from those on an appropriate null model; it merely \term{redescribes} what the null already predicts if they aren't. The near-miss ratio redescribes — a common occupational hazard for network descriptors. The question is whether occupation probability accommodates.

The per-vertex $\occ(v)$ is more promising, but the critical question is whether it tells biologists something that vertex degree alone doesn't. If $\occ(v)$ residuals from degree-predicted values correlate with functional role (synaptic activity in dendrites, regulatory control in SIMs), then the polynomial is doing genuine epistemic work.

The SIM case (Section~3.7) is suggestive: the TF's vanishing $\occ(v)$ maps onto a real causal asymmetry. The phylogeny case is weaker; no biological process \enquote{cares about} independent sets in an evolutionary tree. Whether $\occ(v)$ passes the accommodation test is the open question that determines the framework's biological value.

For trees, this richness comes at manageable cost. The companion paper's algorithm computes $\indpoly$ exactly in $O(n \cdot \alpha^2)$ time using arbitrary-precision arithmetic \citep{reynolds2026erdos}. On general graphs, computing the independence polynomial is \#P-hard; the tree case is tractable precisely because there are no cycles.

As a descriptor, the polynomial functions as a topological fingerprint.\footnote{The independence polynomial is not a complete graph invariant: non-isomorphic trees can share the same polynomial. It is, nonetheless, finer than the degree sequence alone.} Two trees with the same degree sequence can have different independence polynomials, and vice versa. This makes $\indpoly$ complementary to existing descriptors rather than redundant. Because many biological networks have acyclic components (regulatory cascades \citep{alon2007network} and single-input modules are trees by construction), the structural results proved for exact trees apply directly to those subgraphs, though extending them to full networks with cycles remains open (see Section~\ref{sec:limitations}).

\subsection{Cross-disciplinary connections}

The independence polynomial sits at a crossroads of three disciplines. In statistical physics, $\indpoly$ is the partition function of the \term{hard-core lattice gas} at fugacity $\lambda = x$. The structural constraints proved in the companion paper (the edge bound $\occ(u) + \occ(v) < 2/3$, the Hub Exclusion Lemma) are statements about this model on tree-structured lattices \citep{galvin2004weighted, scott2005repulsive, weitz2006counting}.

In chemical graph theory, the single evaluation $I(G;\, 1)$ counts all independent sets of~$G$ and is called the \term{Merrifield--Simmons index} $\sigma(G)$. \textcite{hosoya1971topological} introduced the closely related topological index~$Z(G)$ for matchings, and \textcite{merrifield1989topological} studied $\sigma(G)$ as a molecular descriptor for acyclic hydrocarbons. Both indices have been investigated as predictors of thermodynamic properties in chemical graph theory \citep{wagner2010maxima}, though the full polynomial $\indpoly$ carries richer information than any single evaluation.

This paper extends the framework to biology. The structural results (hub exclusion, the edge bound, near-miss asymptotics) are new contributions from the companion paper \citep{reynolds2026erdos}, and they apply across all three domains. What changes between domains is the interpretation: fugacity in physics, molecular topology in chemistry, branching architecture in biology.

\subsection{Limitations and open questions}\label{sec:limitations}

The unimodality conjecture for tree independence polynomials remains unproved. It has been verified exhaustively for all 1\,198\,738\,056 abstract trees on at most 27~vertices and confirmed for all 27~biological trees in this study, but a general proof is still missing. The structural results themselves (Hub Exclusion and the edge bound) are proved unconditionally and don't depend on the conjecture, but the conjecture's open status means the bell-curve shape visible in Figure~\ref{fig:polyshapes} can't yet be asserted as universal.

Trees are clean; biology is not. Dendritic arbors have gap junctions; phylogenies have reticulation events; regulatory networks have feedback loops. Extending the present results to graphs of bounded treewidth is a natural next step. The algorithm generalizes directly via dynamic programming on tree decompositions, but the structural results (Hub Exclusion, the edge bound) rely on tree-specific arguments and don't obviously carry over.

Several empirical directions are open. The occupation probabilities $\occ(v)$ computed here (Section~3.5) show a clear inverse relationship with vertex degree, but for phylogenies the biological interpretation remains thin: that $\occ(v) = 0.023$ for a polytomy root doesn't correspond to any obvious evolutionary process, and saying that the vertex is \enquote{rarely included in a random independent set} adds no phylogenetic insight beyond what degree alone tells us. A systematic comparison with established centrality measures (betweenness, closeness, eigenvector centrality) across large datasets would clarify whether $\occ(v)$ captures genuinely new structural information.

A full analysis of NeuroMorpho.Org's 260\,000+ reconstructions \citep{ascoli2007neuromorpho} is computationally feasible and would strengthen the empirical base considerably. The null-model result (Section~3.6), that biological $\nm$ is indistinguishable from random trees of the same size, suggests that if the independence polynomial captures biologically meaningful structure, it's through the per-vertex probabilities rather than the global near-miss ratio.

An open question with direct biological relevance is whether independence polynomial features can discriminate between generating processes at fixed~$n$; for instance, distinguishing Yule-process \citep{yule1925mathematical} phylogenies from coalescent trees, or random branching dendrites from reconstructed arbors. This is the accommodation test (Section~5.1) applied at finer grain: if $\indpoly$ can distinguish generating processes, it accommodates a biological distinction; if not, it fails at that level of resolution.

Finally, extending the analysis from exact trees to graphs of bounded treewidth would bring the framework closer to the messy reality of biological networks with their feedback loops and gap junctions.

\section{Conclusion}

I applied the independence polynomial framework developed in the companion paper \citep{reynolds2026erdos} to 27 real biological trees drawn from neuronal reconstructions and published phylogenies, plus 13~\textit{E.\@ coli} transcription regulatory SIMs. All specimens are unimodal and log-concave, consistent with the exhaustive verification of every abstract tree on at most 27~vertices but now confirmed on trees two orders of magnitude larger.

Three structural results carry biological content. First, the Hub Exclusion Lemma predicts that high-degree vertices (soma, polytomies, transcription factors) can't survive in 1-Private maximal independent configurations; their peripheral neighbours must. Per-vertex occupation probabilities confirm this quantitatively: leaves cluster near $\occ(v) \approx 0.5$; high-degree vertices fall well below $1/3$. Second, the hard-core edge bound $\occ(u) + \occ(v) < 2/3$ constrains co-occurrence along every edge. Third, the near-miss ratio $\nm(T)$ rises toward~1 with tree size ($\rho = 0.973$, $p < 0.001$), but a null-model comparison shows this convergence is a generic property of large trees, not a signature of biological architecture.

It's not the global shape of the independence polynomial that carries biological information~-- any large tree, biological or random, converges toward the same bell curve. It's the vertex-level decomposition: which nodes carry high $\occ(v)$ and which are structurally expendable. If the independence polynomial earns a place alongside degree distribution and centrality measures as a network descriptor, it will be through this local structure, not the global envelope.

\section*{Data and code availability}

The SWC and Newick input files, computed independence polynomials, and all analysis scripts (\texttt{indpoly.py}, \texttt{bio\_trees.py}) are available at \url{https://github.com/brettrey/tree-indpoly}. NeuroMorpho.Org accession identifiers and Open Tree of Life clade queries are listed in Table~\ref{tab:results}.

\section*{Acknowledgements}

% TODO: Replace with the actual models used on this project and their current versions.
% Check model names: frontier models change frequently (e.g., GPT-4 → GPT-4o → o3).
This paper was drafted with the assistance of large language models. All content has been reviewed and revised by the author, who takes full responsibility for the final text (and for any remaining stubbornness in the prose).

\newpage
\printbibliography

\end{document}
