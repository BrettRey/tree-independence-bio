\documentclass[12pt]{article}

% House style preamble
% !TEX TS-program = xelatex
% File: preamble.tex
% Purpose: Brett Reynolds house style LaTeX preamble
% Version: 2.0.0 (December 2025 typography redesign)
%
% Usage: % !TEX TS-program = xelatex
% File: preamble.tex
% Purpose: Brett Reynolds house style LaTeX preamble
% Version: 2.0.0 (December 2025 typography redesign)
%
% Usage: % !TEX TS-program = xelatex
% File: preamble.tex
% Purpose: Brett Reynolds house style LaTeX preamble
% Version: 2.0.0 (December 2025 typography redesign)
%
% Usage: \input{.house-style/preamble.tex} in main document
%
% Compiler: XeLaTeX (default) or LuaLaTeX
%           NOT pdfLaTeX - requires fontspec for fonts

% ===========================
% FONTS
% ===========================
\usepackage{fontspec}

\setmainfont{EB Garamond}[
  Numbers=OldStyle,
  Ligatures=TeX,
]

% IPA fallback font
\newfontfamily\ipafont{Charis SIL}
\newcommand{\ipa}[1]{{\ipafont #1}}

% Lining figures when needed (tables, years in isolation)
\providecommand{\liningnums}[1]{{\addfontfeatures{Numbers=Lining}#1}}

% Monospace
\setmonofont{Inconsolata}[Scale=MatchLowercase]

% ===========================
% PAGE LAYOUT
% ===========================
\usepackage[
  letterpaper,
  inner=1.25in,
  outer=1in,
  top=1in,
  bottom=1.25in,
  marginparwidth=0.6in,
]{geometry}

\usepackage[british]{babel}
\usepackage[final]{microtype}

% ===========================
% HEADINGS
% ===========================
\usepackage{titlesec}

% Section: small caps, number in margin
\titleformat{\section}
  {\normalfont\scshape}
  {\llap{\thesection\quad}}
  {0pt}
  {}

% Subsection: small caps sentence case
\titleformat{\subsection}
  {\normalfont\scshape}
  {\thesubsection\quad}
  {0pt}
  {}

% Subsubsection: upright
\titleformat{\subsubsection}
  {\normalfont}
  {\thesubsubsection\quad}
  {0pt}
  {}

% Spacing
\titlespacing*{\section}{0pt}{2ex plus 1ex minus .2ex}{1ex plus .2ex}
\titlespacing*{\subsection}{0pt}{1.5ex plus 1ex minus .2ex}{0.5ex plus .2ex}
\titlespacing*{\subsubsection}{0pt}{1ex plus 0.5ex minus .1ex}{0.3ex plus .1ex}

% ===========================
% RUNNING HEADS
% ===========================
\usepackage{fancyhdr}
\pagestyle{fancy}
\fancyhf{}
\fancyhead[LE]{\small\scshape\leftmark}
\fancyhead[RO]{\small\scshape\@title}
\fancyfoot[LE,RO]{\thepage}
\renewcommand{\headrulewidth}{0pt}

% For oneside documents (most papers):
\fancyhead[L]{\small\scshape\leftmark}
\fancyhead[R]{\small\thepage}
\fancyfoot{}

% ===========================
% COLORS & HYPERLINKS
% ===========================
\usepackage{xcolor}
\definecolor{linkmaroon}{RGB}{128, 0, 32}

\usepackage{hyperref}
\hypersetup{
  colorlinks=true,
  linkcolor=linkmaroon,
  citecolor=linkmaroon,
  urlcolor=linkmaroon,
  pdfauthor={Brett Reynolds},
}

\usepackage{orcidlink}

% ===========================
% QUOTATIONS
% ===========================
\usepackage[style=american]{csquotes} % Double quotes outer (Oxford spelling kept via babel)
\MakeOuterQuote{"}

% ===========================
% SEMANTIC MACROS
% ===========================
% Terms (linguistic concepts being introduced/defined)
\newcommand{\term}[1]{\textsc{#1}}

% Mentions (words/expressions under discussion)
\newcommand{\mention}[1]{\textit{#1}}

% Mentions in headings (angle brackets instead of italics)
% Robust: headings are moving arguments (written to .toc/.aux)
\DeclareRobustCommand{\mentionhead}[1]{⟨\textup{#1}⟩}

% Object language (cited forms, foreign words)
\newcommand{\olang}[1]{\textit{#1}}

% Small-caps abbreviations for glosses
\newcommand{\abbr}[1]{\textsc{#1}}

% Cross-linguistic subscript marker
\usepackage{marvosym}
\newcommand{\crossmark}{\textsubscript{\Cross}}

% ===========================
% MATHS AND SYMBOLS
% ===========================
\usepackage{amsmath,amssymb}

% ===========================
% LINGUISTIC EXAMPLES
% ===========================
\usepackage{langsci-gb4e}
\makeatletter
\@ifundefined{noautomath}{}{\noautomath}
\makeatother

% Judgement markers
\newcommand{\ungram}[1]{*\!#1}
\newcommand{\marg}[1]{?\!#1}
\newcommand{\odd}[1]{\#\!#1}

% ===========================
% LISTS
% ===========================
\usepackage{enumitem}
\setlist{nosep, leftmargin=*}
\setlist[enumerate]{label=\arabic*.}
\setlist[itemize]{label=--}

% ===========================
% TABLES & FIGURES
% ===========================
\usepackage{booktabs}
\usepackage{array}
\usepackage{graphicx}
\graphicspath{{figures/}}

% ===========================
% BIBLIOGRAPHY
% ===========================
\usepackage[backend=biber,style=apa,natbib=true,doi=true,isbn=false,url=true]{biblatex}
\addbibresource{references.bib}
\IfFileExists{references-local.bib}{\addbibresource{references-local.bib}}{}

% Possessive citation: Author's (Year)
\newcommand{\posscite}[1]{\citeauthor{#1}'s (\citeyear{#1})}

% ===========================
% UTILITIES
% ===========================
\usepackage{xspace}
\newcommand{\eg}{e.g.,\xspace}
\newcommand{\ie}{i.e.,\xspace}
\newcommand{\etc}{etc.\xspace}
 in main document
%
% Compiler: XeLaTeX (default) or LuaLaTeX
%           NOT pdfLaTeX - requires fontspec for fonts

% ===========================
% FONTS
% ===========================
\usepackage{fontspec}

\setmainfont{EB Garamond}[
  Numbers=OldStyle,
  Ligatures=TeX,
]

% IPA fallback font
\newfontfamily\ipafont{Charis SIL}
\newcommand{\ipa}[1]{{\ipafont #1}}

% Lining figures when needed (tables, years in isolation)
\providecommand{\liningnums}[1]{{\addfontfeatures{Numbers=Lining}#1}}

% Monospace
\setmonofont{Inconsolata}[Scale=MatchLowercase]

% ===========================
% PAGE LAYOUT
% ===========================
\usepackage[
  letterpaper,
  inner=1.25in,
  outer=1in,
  top=1in,
  bottom=1.25in,
  marginparwidth=0.6in,
]{geometry}

\usepackage[british]{babel}
\usepackage[final]{microtype}

% ===========================
% HEADINGS
% ===========================
\usepackage{titlesec}

% Section: small caps, number in margin
\titleformat{\section}
  {\normalfont\scshape}
  {\llap{\thesection\quad}}
  {0pt}
  {}

% Subsection: small caps sentence case
\titleformat{\subsection}
  {\normalfont\scshape}
  {\thesubsection\quad}
  {0pt}
  {}

% Subsubsection: upright
\titleformat{\subsubsection}
  {\normalfont}
  {\thesubsubsection\quad}
  {0pt}
  {}

% Spacing
\titlespacing*{\section}{0pt}{2ex plus 1ex minus .2ex}{1ex plus .2ex}
\titlespacing*{\subsection}{0pt}{1.5ex plus 1ex minus .2ex}{0.5ex plus .2ex}
\titlespacing*{\subsubsection}{0pt}{1ex plus 0.5ex minus .1ex}{0.3ex plus .1ex}

% ===========================
% RUNNING HEADS
% ===========================
\usepackage{fancyhdr}
\pagestyle{fancy}
\fancyhf{}
\fancyhead[LE]{\small\scshape\leftmark}
\fancyhead[RO]{\small\scshape\@title}
\fancyfoot[LE,RO]{\thepage}
\renewcommand{\headrulewidth}{0pt}

% For oneside documents (most papers):
\fancyhead[L]{\small\scshape\leftmark}
\fancyhead[R]{\small\thepage}
\fancyfoot{}

% ===========================
% COLORS & HYPERLINKS
% ===========================
\usepackage{xcolor}
\definecolor{linkmaroon}{RGB}{128, 0, 32}

\usepackage{hyperref}
\hypersetup{
  colorlinks=true,
  linkcolor=linkmaroon,
  citecolor=linkmaroon,
  urlcolor=linkmaroon,
  pdfauthor={Brett Reynolds},
}

\usepackage{orcidlink}

% ===========================
% QUOTATIONS
% ===========================
\usepackage[style=american]{csquotes} % Double quotes outer (Oxford spelling kept via babel)
\MakeOuterQuote{"}

% ===========================
% SEMANTIC MACROS
% ===========================
% Terms (linguistic concepts being introduced/defined)
\newcommand{\term}[1]{\textsc{#1}}

% Mentions (words/expressions under discussion)
\newcommand{\mention}[1]{\textit{#1}}

% Mentions in headings (angle brackets instead of italics)
% Robust: headings are moving arguments (written to .toc/.aux)
\DeclareRobustCommand{\mentionhead}[1]{⟨\textup{#1}⟩}

% Object language (cited forms, foreign words)
\newcommand{\olang}[1]{\textit{#1}}

% Small-caps abbreviations for glosses
\newcommand{\abbr}[1]{\textsc{#1}}

% Cross-linguistic subscript marker
\usepackage{marvosym}
\newcommand{\crossmark}{\textsubscript{\Cross}}

% ===========================
% MATHS AND SYMBOLS
% ===========================
\usepackage{amsmath,amssymb}

% ===========================
% LINGUISTIC EXAMPLES
% ===========================
\usepackage{langsci-gb4e}
\makeatletter
\@ifundefined{noautomath}{}{\noautomath}
\makeatother

% Judgement markers
\newcommand{\ungram}[1]{*\!#1}
\newcommand{\marg}[1]{?\!#1}
\newcommand{\odd}[1]{\#\!#1}

% ===========================
% LISTS
% ===========================
\usepackage{enumitem}
\setlist{nosep, leftmargin=*}
\setlist[enumerate]{label=\arabic*.}
\setlist[itemize]{label=--}

% ===========================
% TABLES & FIGURES
% ===========================
\usepackage{booktabs}
\usepackage{array}
\usepackage{graphicx}
\graphicspath{{figures/}}

% ===========================
% BIBLIOGRAPHY
% ===========================
\usepackage[backend=biber,style=apa,natbib=true,doi=true,isbn=false,url=true]{biblatex}
\addbibresource{references.bib}
\IfFileExists{references-local.bib}{\addbibresource{references-local.bib}}{}

% Possessive citation: Author's (Year)
\newcommand{\posscite}[1]{\citeauthor{#1}'s (\citeyear{#1})}

% ===========================
% UTILITIES
% ===========================
\usepackage{xspace}
\newcommand{\eg}{e.g.,\xspace}
\newcommand{\ie}{i.e.,\xspace}
\newcommand{\etc}{etc.\xspace}
 in main document
%
% Compiler: XeLaTeX (default) or LuaLaTeX
%           NOT pdfLaTeX - requires fontspec for fonts

% ===========================
% FONTS
% ===========================
\usepackage{fontspec}

\setmainfont{EB Garamond}[
  Numbers=OldStyle,
  Ligatures=TeX,
]

% IPA fallback font
\newfontfamily\ipafont{Charis SIL}
\newcommand{\ipa}[1]{{\ipafont #1}}

% Lining figures when needed (tables, years in isolation)
\providecommand{\liningnums}[1]{{\addfontfeatures{Numbers=Lining}#1}}

% Monospace
\setmonofont{Inconsolata}[Scale=MatchLowercase]

% ===========================
% PAGE LAYOUT
% ===========================
\usepackage[
  letterpaper,
  inner=1.25in,
  outer=1in,
  top=1in,
  bottom=1.25in,
  marginparwidth=0.6in,
]{geometry}

\usepackage[british]{babel}
\usepackage[final]{microtype}

% ===========================
% HEADINGS
% ===========================
\usepackage{titlesec}

% Section: small caps, number in margin
\titleformat{\section}
  {\normalfont\scshape}
  {\llap{\thesection\quad}}
  {0pt}
  {}

% Subsection: small caps sentence case
\titleformat{\subsection}
  {\normalfont\scshape}
  {\thesubsection\quad}
  {0pt}
  {}

% Subsubsection: upright
\titleformat{\subsubsection}
  {\normalfont}
  {\thesubsubsection\quad}
  {0pt}
  {}

% Spacing
\titlespacing*{\section}{0pt}{2ex plus 1ex minus .2ex}{1ex plus .2ex}
\titlespacing*{\subsection}{0pt}{1.5ex plus 1ex minus .2ex}{0.5ex plus .2ex}
\titlespacing*{\subsubsection}{0pt}{1ex plus 0.5ex minus .1ex}{0.3ex plus .1ex}

% ===========================
% RUNNING HEADS
% ===========================
\usepackage{fancyhdr}
\pagestyle{fancy}
\fancyhf{}
\fancyhead[LE]{\small\scshape\leftmark}
\fancyhead[RO]{\small\scshape\@title}
\fancyfoot[LE,RO]{\thepage}
\renewcommand{\headrulewidth}{0pt}

% For oneside documents (most papers):
\fancyhead[L]{\small\scshape\leftmark}
\fancyhead[R]{\small\thepage}
\fancyfoot{}

% ===========================
% COLORS & HYPERLINKS
% ===========================
\usepackage{xcolor}
\definecolor{linkmaroon}{RGB}{128, 0, 32}

\usepackage{hyperref}
\hypersetup{
  colorlinks=true,
  linkcolor=linkmaroon,
  citecolor=linkmaroon,
  urlcolor=linkmaroon,
  pdfauthor={Brett Reynolds},
}

\usepackage{orcidlink}

% ===========================
% QUOTATIONS
% ===========================
\usepackage[style=american]{csquotes} % Double quotes outer (Oxford spelling kept via babel)
\MakeOuterQuote{"}

% ===========================
% SEMANTIC MACROS
% ===========================
% Terms (linguistic concepts being introduced/defined)
\newcommand{\term}[1]{\textsc{#1}}

% Mentions (words/expressions under discussion)
\newcommand{\mention}[1]{\textit{#1}}

% Mentions in headings (angle brackets instead of italics)
% Robust: headings are moving arguments (written to .toc/.aux)
\DeclareRobustCommand{\mentionhead}[1]{⟨\textup{#1}⟩}

% Object language (cited forms, foreign words)
\newcommand{\olang}[1]{\textit{#1}}

% Small-caps abbreviations for glosses
\newcommand{\abbr}[1]{\textsc{#1}}

% Cross-linguistic subscript marker
\usepackage{marvosym}
\newcommand{\crossmark}{\textsubscript{\Cross}}

% ===========================
% MATHS AND SYMBOLS
% ===========================
\usepackage{amsmath,amssymb}

% ===========================
% LINGUISTIC EXAMPLES
% ===========================
\usepackage{langsci-gb4e}
\makeatletter
\@ifundefined{noautomath}{}{\noautomath}
\makeatother

% Judgement markers
\newcommand{\ungram}[1]{*\!#1}
\newcommand{\marg}[1]{?\!#1}
\newcommand{\odd}[1]{\#\!#1}

% ===========================
% LISTS
% ===========================
\usepackage{enumitem}
\setlist{nosep, leftmargin=*}
\setlist[enumerate]{label=\arabic*.}
\setlist[itemize]{label=--}

% ===========================
% TABLES & FIGURES
% ===========================
\usepackage{booktabs}
\usepackage{array}
\usepackage{graphicx}
\graphicspath{{figures/}}

% ===========================
% BIBLIOGRAPHY
% ===========================
\usepackage[backend=biber,style=apa,natbib=true,doi=true,isbn=false,url=true]{biblatex}
\addbibresource{references.bib}
\IfFileExists{references-local.bib}{\addbibresource{references-local.bib}}{}

% Possessive citation: Author's (Year)
\newcommand{\posscite}[1]{\citeauthor{#1}'s (\citeyear{#1})}

% ===========================
% UTILITIES
% ===========================
\usepackage{xspace}
\newcommand{\eg}{e.g.,\xspace}
\newcommand{\ie}{i.e.,\xspace}
\newcommand{\etc}{etc.\xspace}


% Project-specific macros
\newcommand{\indpoly}{I(T;\,x)}
\newcommand{\nm}{\mathrm{nm}}
\newcommand{\occ}{P}
\DeclareMathOperator{\mode}{mode}

% Update PDF metadata
\hypersetup{
  pdftitle={Tree Independence Polynomials and Biological Network Motifs}
}

\title{Tree Independence Polynomials and Biological Network Motifs}
\author{Brett Reynolds \orcidlink{0000-0003-0073-7195}%
\thanks{Contact: \href{mailto:brett.reynolds@humber.ca}{brett.reynolds@humber.ca}}\\
Humber Polytechnic \& University of Toronto}
\date{\today}

\begin{document}
\maketitle

\begin{abstract}
TODO: Write abstract here.
\end{abstract}

\section{Introduction}

Trees are everywhere in biology. Phylogenies record the branching history of species \citep{hinchliff2015synthesis}. Dendritic arbors carry signals from synapse to soma. The backbone of a transcription regulatory network, stripped of its cycles, is a tree \citep{alon2007network}. In each case the branching pattern isn't just a shape; it constrains what the system can do. This paper asks what a single polynomial can reveal about that constraint.

The \term{independence polynomial} of a tree~$T$ counts the ways to select non-adjacent vertices. Call a set of vertices \term{independent} if no two are neighbours, and let $i_k(T)$ be the number of independent sets of size~$k$. The polynomial $\indpoly = \sum_k i_k(T)\, x^k$ packages these counts. \textcite{alavi1987vertex} conjectured that for every tree the coefficient sequence $i_0, i_1, \ldots$ is \term{unimodal}: it rises to a single peak and then falls. This is Erd\H{o}s Problem~993. Despite partial results \citep{chudnovsky2007roots, wagner2010maxima}, the conjecture remains open, though \textcite{reynolds2026erdos} has verified it for all 447\,672\,596 trees on at most 26~vertices.

Why should biologists care about a graph-theoretic polynomial? In a protein interaction network whose local topology is tree-like \citep{milo2002network}, an independent set is a collection of proteins no two of which interact directly~-- a non-interacting module. In a dendritic arbor, it's a set of compartments that don't share a branch point. The shape of $\indpoly$~-- its mode, its width, its \term{near-miss ratio}~$\nm(T)$~-- characterises how the tree's topology distributes combinatorial freedom. The \term{hard-core model} from statistical physics \citep{galvin2004weighted, scott2005repulsive} assigns an occupation probability $\occ(v)$ to each vertex; these probabilities serve as centrality measures that reflect a vertex's role in the tree's independent-set structure.

This paper applies results from a companion study \citep{reynolds2026erdos} to real biological trees. We compute independence polynomials for neuronal reconstructions from NeuroMorpho.Org \citep{ascoli2007neuromorpho} and phylogenetic trees from the Open Tree of Life \citep{hinchliff2015synthesis}, confirming unimodality and log-concavity across all specimens. We interpret three results in biological terms: the Hub Exclusion Lemma (high-degree vertices are excluded from maximal independent configurations), the hard-core edge bound ($\occ(u) + \occ(v) < 2/3$ for adjacent vertices), and the near-miss ratio as a measure of how close a tree's topology sits to the combinatorial boundary.

\section{Mathematical background}

\subsection{Independence polynomials and unimodality}

A \term{graph} $G = (V, E)$ is a set of \term{vertices}~$V$ and a set of \term{edges}~$E$, each edge joining two vertices. A \term{tree} is a connected graph with no cycles. An \term{independent set} in~$G$ is a subset $S \subseteq V$ in which no two vertices share an edge. The \term{independence number} $\alpha(G)$ is the size of a largest independent set.

Write $i_k(G)$ for the number of independent sets of size~$k$, with $i_0 = 1$ (the empty set). The \term{independence polynomial} is
\[
  \indpoly \;=\; \sum_{k=0}^{\alpha(T)} i_k(T)\, x^k.
\]
This polynomial packages all the independent-set counts into a single object. Its coefficient sequence $i_0, i_1, \ldots, i_{\alpha}$ is \term{unimodal} if it rises to a single peak and then falls: there's an index~$p$ (the \term{mode}) with $i_0 \le i_1 \le \cdots \le i_p \ge i_{p+1} \ge \cdots \ge i_\alpha$.

\textcite{alavi1987vertex} conjectured that every tree's independence polynomial is unimodal. Despite decades of partial results~-- including special cases such as paths, caterpillars, and spiders~-- the conjecture remains open \citep{wagner2010maxima}. \textcite{reynolds2026erdos} verified it computationally for all 447\,672\,596 trees on at most 26~vertices.

A stronger property is \term{log-concavity}: $i_k^2 \ge i_{k-1}\, i_{k+1}$ for all~$k$. Log-concavity implies unimodality, but not conversely. \textcite{chudnovsky2007roots} proved that independence polynomials of claw-free graphs are real-rooted, which forces log-concavity for that class. Trees aren't generally claw-free, however, and at $n = 26$ two trees have log-concavity failures~-- yet both remain unimodal \citep{reynolds2026erdos}.

\subsection{The hard-core model}

The \term{hard-core model} at fugacity $\lambda = 1$ places a uniform distribution over all independent sets of~$T$. Each IS is equally likely. This model originates in statistical physics, where it describes non-overlapping particle occupation on a lattice; here the lattice is a tree \citep{galvin2004weighted, scott2005repulsive}.

Under this distribution, the \term{occupation probability} of a vertex~$v$ is $\occ(v) = \Pr[v \in S]$, the probability that~$v$ belongs to a uniformly random independent set. These probabilities serve as a kind of centrality measure: $\occ(v)$ reflects how strongly the tree's topology favours including~$v$ in an independent configuration.

The expected size of a random independent set is
\[
  \mu \;=\; \sum_{v \in V} \occ(v) \;=\; \frac{I'(T;\, 1)}{I(T;\, 1)},
\]
where $I'$ is the derivative. This connects the polynomial's analytic properties to a statistical quantity: $\mu$ is the mean of the coefficient distribution, weighted by the $i_k$.

The companion paper proves a structural constraint on adjacent occupation probabilities. For any edge $uv$ in a tree on $n \ge 3$ vertices,
\[
  \occ(u) + \occ(v) \;<\; \frac{2}{3}.
\]
This \term{edge bound} has an immediate consequence: the set $\{v : \occ(v) > 1/3\}$ is automatically an independent set, since two adjacent vertices can't both exceed~$1/3$. High-probability vertices repel each other. In biological terms, if two proteins interact directly, they can't both be highly likely members of a random non-interacting module.

\subsection{Structural reduction: the 1-Private framework}

The companion paper develops a structural reduction that isolates the combinatorially hardest trees. The key concept is that of a \term{private neighbour}. Given a maximal independent set~$S$, a non-$S$ vertex~$w$ is \term{private} to $u \in S$ if $u$ is the only $S$-member adjacent to~$w$. A maximal independent set is \term{1-Private} if every vertex in~$S$ has at most one private neighbour.

Write $d_{\mathrm{leaf}}(v)$ for the number of leaf-children of vertex~$v$. The \term{Hub Exclusion Lemma} constrains how high-degree vertices interact with 1-Private independent sets.

\medskip
\noindent\textbf{Hub Exclusion Lemma} \citep{reynolds2026erdos}.
\textit{If $d_{\mathrm{leaf}}(v) \ge 2$ and $S$ is a 1-Private maximal independent set, then $v \notin S$, and all leaf-children of~$v$ lie in~$S$.}
\medskip

The logic is direct: if~$v$ were in~$S$, each of its $\ge 2$ leaf-children would be private to~$v$, violating the 1-Private bound. Since $v \notin S$, domination forces every leaf-child into~$S$.

The \term{Transfer Lemma} ensures this reduction propagates. Pruning a hub~$v$ together with its leaf-children produces a residual tree (or forest)~$T'$, and the restriction $S' = S \cap V(T')$ remains a 1-Private maximal independent set in~$T'$. Iterating over all hubs with $d_{\mathrm{leaf}} \ge 2$ terminates at a residual in which every vertex has at most one leaf-child~-- a structurally simpler class of trees.

The reduction has practical implications for biological networks. Hub vertices in protein interaction networks~-- proteins with many single-degree interaction partners~-- are automatically excluded from 1-Private configurations while their partners are included. This parallels the empirical observation that hub proteins tend to be regulated rather than regulators.

Finally, the \term{near-miss ratio} $\nm(T)$ quantifies how close a tree sits to the unimodality boundary. Let $j_0$ be the first index where $i_{j_0} > i_{j_0+1}$ (the first strict descent in the coefficient sequence). Then
\[
  \nm(T) \;=\; \max_{j > j_0} \frac{i_{j+1}(T)}{i_j(T)}.
\]
A value exceeding~1 would be a unimodality violation; values near~1 indicate the tree's topology nearly permits one. The companion paper shows that for trees obtained by attaching $s$~pendant leaves to a fixed hub, $\nm(s) = 1 - C/s + O(1/s^2)$ with $C \in [4, 8)$. The margin shrinks as~$s$ grows but doesn't vanish.

\section{Biological trees as independence structures}

\subsection{Neuronal dendritic arbors}

Dendritic trees are literal trees. The standard SWC file format used by NeuroMorpho.Org \citep{ascoli2007neuromorpho} encodes each neuronal reconstruction as a rooted tree: every compartment (a short cylinder of membrane) is a vertex, and edges record the parent-child relationship along the dendrite. The graph has no cycles by construction.

An independent set in a dendritic tree is a collection of compartments no two of which share a branch segment~-- non-adjacent functional units. The Hub Exclusion Lemma makes a structural prediction here: high-branching points (the soma, primary bifurcations) have multiple leaf-children and so are excluded from any 1-Private maximal independent set, while their terminal dendrites are automatically included. This matches biological intuition. Terminal arbors are the \enquote{active} sites~-- they receive synaptic input and initiate local dendritic computation~-- whereas branch points serve a structural, routing role.

The hard-core edge bound reinforces the picture. For any parent-child pair along the dendrite, $\occ(\text{parent}) + \occ(\text{child}) < 2/3$: the occupation probabilities of adjacent compartments can't both be high. The independence polynomial thus quantifies a topological constraint that biologists already sense informally: branching architecture partitions the arbor into zones of combinatorial freedom concentrated at the periphery.

\subsection{Phylogenetic trees}

Phylogenies are trees by definition. Tips represent extant species; internal nodes represent ancestral divergence events. The Open Tree of Life \citep{hinchliff2015synthesis} assembles published phylogenies into a single supertree spanning over two million tips~-- a dataset that is both enormous and, at every local neighbourhood, a tree.

The independence polynomial $\indpoly$ encodes the combinatorial structure of evolutionary branching. Each coefficient $i_k$ counts the number of ways to select $k$ non-adjacent nodes in the phylogeny. The shape of the polynomial~-- its mode, width, near-miss ratio~-- characterises properties of the branching pattern that aren't captured by standard measures like balance or gamma statistics.

\term{Polytomies} (unresolved branching events where an ancestor splits into three or more descendants simultaneously) are the hubs where Hub Exclusion applies. A polytomy with $d_{\mathrm{leaf}} \ge 2$ descendant tips is excluded from every 1-Private maximal independent set; its tip descendants are included instead. Tip species dominate the independent-set structure, while deep ancestors with many descendants don't participate. The near-miss ratio $\nm(T)$ characterises how \enquote{robust} the phylogeny's combinatorial structure is: values near~1 indicate the branching pattern sits close to the unimodality boundary, suggesting a topology where adding or removing a single edge could shift the polynomial's shape.

\subsection{Transcription regulatory motifs}

\term{Single-Input Modules} (SIMs) are among the most common motifs in transcription regulatory networks: one transcription factor (TF) regulates $N$~target genes \citep{alon2007network, milo2002network}. Topologically, a SIM is a star~-- a tree with one hub and $N$~leaves. This makes it the simplest non-trivial case for the independence polynomial framework.

Hub Exclusion applies directly. The TF is the hub vertex with $d_{\mathrm{leaf}} = N \ge 2$ leaf-children, so it's excluded from every 1-Private maximal independent set; the targets are all included. This mirrors the biological logic of a SIM: the target genes can be independently active (expressed or not), but the regulator's state constrains the whole module. The regulator isn't \enquote{free} in the way the targets are.

The hard-core edge bound $\occ(\mathrm{TF}) + \occ(\mathrm{target}_i) < 2/3$ constrains co-occurrence between the regulator and any one of its targets. In the hard-core model, the TF has low occupation probability (it belongs to few independent sets relative to the total) while each target has high occupation probability. For a star $K_{1,N}$, the TF's occupation probability is $1/(2^N + 1)$, which drops toward zero as $N$ grows. The targets collectively dominate the independent-set structure~-- exactly the asymmetry that makes SIMs function as regulatory amplifiers.

\section{Computational results}

\subsection{Data and methods}

We analysed 26 biological trees drawn from two sources. From NeuroMorpho.Org \citep{ascoli2007neuromorpho} we obtained 15 neuronal reconstructions in SWC format, spanning five species (Drosophila, macaque, human, mouse, rat) and several cell types (pyramidal cells, Purkinje cells, interneurons, principal cells). Tree sizes range from $n = 113$ (Drosophila ddaC) to $n = 17{,}992$ (human pyramidal). From the Open Tree of Life \citep{hinchliff2015synthesis} we extracted 11 phylogenetic subtrees in Newick format, ranging from $n = 46$ (Homininae) to $n = 1{,}387$ (Cetacea).

Independence polynomials were computed exactly using arbitrary-precision integer arithmetic via \texttt{indpoly.py}, the algorithm described in the companion paper \citep{reynolds2026erdos}. Format conversion from SWC and Newick files used \texttt{bio\_trees.py}. Runtime was sub-second for trees with $n < 1{,}000$ and approximately 19~minutes for the largest tree ($n = 17{,}992$). Table~\ref{tab:results} summarises the results for all 26~trees.

\begin{table}[htbp]
\centering
\caption{Independence polynomial properties of biological trees. $n$~= number of nodes, $\alpha$~= independence number, mode~= index of largest coefficient, $\mathrm{nm}$~= near-miss ratio.}
\label{tab:results}
\small
\begin{tabular}{@{}llrrrr@{}}
\toprule
Tree & Species & $n$ & $\alpha$ & mode & $\mathrm{nm}$ \\
\midrule
\multicolumn{6}{@{}l}{\textit{Neuronal arbors (NeuroMorpho.org)}} \\
\addlinespace[2pt]
  Drosophila ddaC (A5) & Drosophila & 113 & 58 & 32 & 0.8478 \\
  Drosophila ddaC (A6) & Drosophila & 166 & 83 & 46 & 0.9317 \\
  Monkey L3 pyramidal (041) & Macaque & 1,038 & 525 & 290 & 0.9813 \\
  Monkey L3 pyramidal (001) & Macaque & 1,274 & 646 & 356 & 0.9873 \\
  Human aspiny interneuron & Human & 1,498 & 752 & 416 & 0.9865 \\
  Mouse Purkinje (P35) & Mouse & 1,716 & 891 & 491 & 0.9896 \\
  Rat interneuron & Rat & 1,941 & 978 & 540 & 0.9927 \\
  Mouse principal cell & Mouse & 2,212 & 1,108 & 612 & 0.9944 \\
  Rat interneuron (IDC) & Rat & 2,774 & 1,392 & 770 & 0.9930 \\
  Rat pyramidal (n419) & Rat & 4,229 & 2,122 & 1,174 & 0.9958 \\
  Mouse Purkinje (P43) & Mouse & 4,156 & 2,126 & 1,172 & 0.9954 \\
  Rat basket interneuron & Rat & 6,378 & 3,199 & 1,767 & 0.9970 \\
  Rat interneuron (A3) & Rat & 7,649 & 3,843 & 2,123 & 0.9981 \\
  Human pyramidal (06) & Human & 10,451 & 5,236 & 2,895 & 0.9980 \\
  Human pyramidal (05) & Human & 17,992 & 9,011 & 4,981 & 0.9990 \\
\addlinespace[6pt]
\multicolumn{6}{@{}l}{\textit{Phylogenetic trees (Open Tree of Life)}} \\
\addlinespace[2pt]
  Homininae & --- & 46 & 28 & 15 & 0.7700 \\
  Delphinidae & --- & 107 & 71 & 36 & 0.9184 \\
  Felidae & --- & 144 & 100 & 52 & 0.9479 \\
  Salamandridae & --- & 157 & 105 & 54 & 0.9423 \\
  Mustelidae & --- & 188 & 122 & 64 & 0.9488 \\
  Canidae & --- & 367 & 241 & 125 & 0.9812 \\
  Equidae & --- & 372 & 223 & 117 & 0.9612 \\
  Accipitridae & --- & 884 & 601 & 311 & 0.9857 \\
  Papilionidae & --- & 1,170 & 929 & 472 & 0.9917 \\
  Primates & --- & 1,333 & 872 & 452 & 0.9941 \\
  Cetacea & --- & 1,387 & 971 & 498 & 0.9911 \\
  Aves & --- & 30,310 & 20,357 & 10,532 & 0.9997 \\
\bottomrule
\end{tabular}
\end{table}

\subsection{Unimodality and log-concavity}

All 26 biological trees have unimodal and log-concave independence polynomials. This extends the exhaustive verification of all abstract trees on $\le 26$ vertices \citep{reynolds2026erdos} to real biological trees of much larger size~-- up to $n = 17{,}992$ vertices. Mode positions cluster near $\alpha/2$ (mode$/\alpha \approx 0.55$ for most trees), consistent with the symmetric bell-curve shape visible in Figure~\ref{fig:polyshapes}. Neither the neuronal arbors nor the phylogenetic trees produced any counterexample to the unimodality conjecture.

\subsection{Near-miss ratio and robustness}

The \term{near-miss ratio} $\nm(T)$ ranges from 0.770 (Homininae, $n = 46$) to 0.999 (human pyramidal, $n = 17{,}992$). Figure~\ref{fig:nmvsn} shows a clear positive correlation between tree size and near-miss ratio, consistent with the asymptotic $1 - C/s$ derived in the companion paper \citep{reynolds2026erdos}.

The pattern admits a clean interpretation. Larger biological trees sit closer to the combinatorial boundary~-- their coefficient sequences nearly fail to be unimodal~-- but they never cross it. We call this the \term{dilution effect}: as $n$~grows, the polynomial acquires more coefficients and the sequence smooths, making near-misses more likely but actual violations no closer. Small trees ($n < 100$) have $\nm$ well below~0.9; large trees ($n > 1{,}000$) all exceed~0.98.

The two data types occupy different regions of Figure~\ref{fig:nmvsn}. Neuronal arbors, with their long chains of degree-2 vertices, tend toward higher~$\nm$ at a given~$n$ than phylogenies of comparable size. Phylogenies with many polytomies have more combinatorial structure per vertex, which keeps $\nm$ slightly lower. Both populations follow the same asymptotic envelope.

\begin{figure}[htbp]
  \centering
  \includegraphics[width=\textwidth]{figures/nm_vs_n.pdf}
  \caption{Near-miss ratio $\nm(T)$ versus tree size~$n$ for 15~neuronal arbors (blue circles) and 11~phylogenetic trees (red triangles). The dashed curve shows the asymptotic envelope $1 - 6/n$. All values lie below~1, confirming unimodality.}
  \label{fig:nmvsn}
\end{figure}

\begin{figure}[htbp]
  \centering
  \includegraphics[width=\textwidth]{figures/poly_shapes.pdf}
  \caption{Normalised independence polynomial coefficients $i_k / \max(i_k)$ versus $k/\alpha$ for four representative trees. The universal unimodal bell-curve shape is evident across both data types and a wide range of tree sizes.}
  \label{fig:polyshapes}
\end{figure}

\subsection{Hub exclusion in biological networks}

The Hub Exclusion Lemma applies wherever a vertex has two or more leaf-children. The two data types differ in where this condition holds. Phylogenies have many hub vertices~-- polytomies where an ancestor diverges into three or more descendants simultaneously~-- so Hub Exclusion is active throughout the tree. Neuronal arbors, by contrast, consist mostly of degree-2 chains (unbranched dendritic segments) with occasional branch points, and hub exclusion applies only at those branch points.

In phylogenies, unresolved radiations (polytomies with $d_{\mathrm{leaf}} \ge 2$) are structurally excluded from maximal 1-Private independent configurations while their tip species are included. This asymmetry is stronger in phylogenies than in neuronal arbors because phylogenies have higher maximum vertex degree. The observation matches biological reality in both systems: peripheral elements~-- terminal dendrites, extant species~-- carry more combinatorial weight than central hubs. The independence polynomial formalises this intuition as a theorem.

\section{Discussion}

\subsection{The independence polynomial as a network descriptor}

Standard network descriptors~-- degree distribution, clustering coefficient, betweenness centrality~-- reduce a graph to a handful of scalars. The independence polynomial retains more. Its full coefficient sequence $i_0, i_1, \ldots, i_\alpha$ encodes how combinatorial freedom is distributed across every possible selection size, not just the mean or maximum. Mode position, near-miss ratio, and per-vertex occupation probabilities $\occ(v)$ each capture a different facet of the tree's branching topology.

For trees, this richness comes at manageable cost. The companion paper's algorithm computes $\indpoly$ exactly in $O(n \cdot \alpha^2)$ time using arbitrary-precision arithmetic \citep{reynolds2026erdos}. On general graphs, computing the independence polynomial is \#P-hard; the tree case is tractable precisely because there are no cycles.

The polynomial functions as a topological fingerprint. Two trees with the same degree sequence can have different independence polynomials, and vice versa. This makes $\indpoly$ complementary to existing descriptors rather than redundant. Because many biological networks are locally tree-like~-- protein interaction networks have tree-like backbones \citep{milo2002network}, regulatory cascades are acyclic~-- the structural results proved for exact trees (hub exclusion, edge bound, near-miss asymptotics) apply at least locally to a broader class of biological networks.

\subsection{Universality: physics, chemistry, biology}

The independence polynomial sits at a crossroads of three disciplines. In statistical physics, $\indpoly$ is the partition function of the \term{hard-core lattice gas} at fugacity~$x$. The structural constraints proved in the companion paper~-- the edge bound $\occ(u) + \occ(v) < 2/3$, the Hub Exclusion Lemma~-- are statements about this model on tree-structured lattices \citep{galvin2004weighted, scott2005repulsive}.

In chemical graph theory, the single evaluation $I(G;\, 1)$ counts all independent sets of~$G$ and is called the \term{Merrifield--Simmons index} $\sigma(G)$. \textcite{hosoya1971topological} introduced the closely related topological index~$Z(G)$ for matchings, and \textcite{merrifield1989topological} studied $\sigma(G)$ as a molecular descriptor for acyclic hydrocarbons. Both indices have been investigated as predictors of thermodynamic properties in chemical graph theory, though the full polynomial $\indpoly$ carries richer information than any single evaluation.

This paper extends the framework to biology. The structural results~-- hub exclusion, the edge bound, near-miss asymptotics~-- are new contributions from the companion paper \citep{reynolds2026erdos}, and they apply across all three domains. What changes between domains is the interpretation: fugacity in physics, molecular topology in chemistry, branching architecture in biology.

\subsection{Limitations and open questions}

The unimodality conjecture for tree independence polynomials remains unproved. It has been verified exhaustively for all 447\,672\,596 abstract trees on at most 26~vertices and confirmed for all 26~biological trees in this study, but a general proof is still missing. The conjecture's status limits the theoretical claims we can make: every structural result conditional on unimodality inherits this gap.

Real biological networks are rarely exact trees. Dendritic arbors have gap junctions; phylogenies have reticulation events; regulatory networks have feedback loops. Extending the present results to graphs of bounded treewidth is a natural next step, since many of the algorithmic and structural properties of trees generalise to that setting.

Several empirical directions are open. The occupation probabilities $\occ(v)$ deserve systematic comparison with existing centrality measures across large datasets. A full analysis of NeuroMorpho.Org's 260\,000+ reconstructions \citep{ascoli2007neuromorpho} is computationally feasible and would strengthen the empirical base considerably. Finally, the connection between $\nm(T)$ and network robustness~-- whether trees near the unimodality boundary are structurally fragile in some biological sense~-- deserves further investigation.

\section{Conclusion}

TODO: Write conclusion.

\section*{Acknowledgements}

% TODO: Replace with the actual models used on this project and their current versions.
% Check model names: frontier models change frequently (e.g., GPT-4 → GPT-4o → o3).
This paper was drafted with the assistance of large language models. All content has been reviewed and revised by the author, who takes full responsibility for the final text.

\newpage
\printbibliography

\end{document}
